% Options for packages loaded elsewhere
\PassOptionsToPackage{unicode}{hyperref}
\PassOptionsToPackage{hyphens}{url}
\documentclass[
]{article}
\usepackage{xcolor}
\usepackage[margin=1in]{geometry}
\usepackage{amsmath,amssymb}
\setcounter{secnumdepth}{-\maxdimen} % remove section numbering
\usepackage{iftex}
\ifPDFTeX
  \usepackage[T1]{fontenc}
  \usepackage[utf8]{inputenc}
  \usepackage{textcomp} % provide euro and other symbols
\else % if luatex or xetex
  \usepackage{unicode-math} % this also loads fontspec
  \defaultfontfeatures{Scale=MatchLowercase}
  \defaultfontfeatures[\rmfamily]{Ligatures=TeX,Scale=1}
\fi
\usepackage{lmodern}
\ifPDFTeX\else
  % xetex/luatex font selection
\fi
% Use upquote if available, for straight quotes in verbatim environments
\IfFileExists{upquote.sty}{\usepackage{upquote}}{}
\IfFileExists{microtype.sty}{% use microtype if available
  \usepackage[]{microtype}
  \UseMicrotypeSet[protrusion]{basicmath} % disable protrusion for tt fonts
}{}
\makeatletter
\@ifundefined{KOMAClassName}{% if non-KOMA class
  \IfFileExists{parskip.sty}{%
    \usepackage{parskip}
  }{% else
    \setlength{\parindent}{0pt}
    \setlength{\parskip}{6pt plus 2pt minus 1pt}}
}{% if KOMA class
  \KOMAoptions{parskip=half}}
\makeatother
\usepackage{color}
\usepackage{fancyvrb}
\newcommand{\VerbBar}{|}
\newcommand{\VERB}{\Verb[commandchars=\\\{\}]}
\DefineVerbatimEnvironment{Highlighting}{Verbatim}{commandchars=\\\{\}}
% Add ',fontsize=\small' for more characters per line
\usepackage{framed}
\definecolor{shadecolor}{RGB}{248,248,248}
\newenvironment{Shaded}{\begin{snugshade}}{\end{snugshade}}
\newcommand{\AlertTok}[1]{\textcolor[rgb]{0.94,0.16,0.16}{#1}}
\newcommand{\AnnotationTok}[1]{\textcolor[rgb]{0.56,0.35,0.01}{\textbf{\textit{#1}}}}
\newcommand{\AttributeTok}[1]{\textcolor[rgb]{0.13,0.29,0.53}{#1}}
\newcommand{\BaseNTok}[1]{\textcolor[rgb]{0.00,0.00,0.81}{#1}}
\newcommand{\BuiltInTok}[1]{#1}
\newcommand{\CharTok}[1]{\textcolor[rgb]{0.31,0.60,0.02}{#1}}
\newcommand{\CommentTok}[1]{\textcolor[rgb]{0.56,0.35,0.01}{\textit{#1}}}
\newcommand{\CommentVarTok}[1]{\textcolor[rgb]{0.56,0.35,0.01}{\textbf{\textit{#1}}}}
\newcommand{\ConstantTok}[1]{\textcolor[rgb]{0.56,0.35,0.01}{#1}}
\newcommand{\ControlFlowTok}[1]{\textcolor[rgb]{0.13,0.29,0.53}{\textbf{#1}}}
\newcommand{\DataTypeTok}[1]{\textcolor[rgb]{0.13,0.29,0.53}{#1}}
\newcommand{\DecValTok}[1]{\textcolor[rgb]{0.00,0.00,0.81}{#1}}
\newcommand{\DocumentationTok}[1]{\textcolor[rgb]{0.56,0.35,0.01}{\textbf{\textit{#1}}}}
\newcommand{\ErrorTok}[1]{\textcolor[rgb]{0.64,0.00,0.00}{\textbf{#1}}}
\newcommand{\ExtensionTok}[1]{#1}
\newcommand{\FloatTok}[1]{\textcolor[rgb]{0.00,0.00,0.81}{#1}}
\newcommand{\FunctionTok}[1]{\textcolor[rgb]{0.13,0.29,0.53}{\textbf{#1}}}
\newcommand{\ImportTok}[1]{#1}
\newcommand{\InformationTok}[1]{\textcolor[rgb]{0.56,0.35,0.01}{\textbf{\textit{#1}}}}
\newcommand{\KeywordTok}[1]{\textcolor[rgb]{0.13,0.29,0.53}{\textbf{#1}}}
\newcommand{\NormalTok}[1]{#1}
\newcommand{\OperatorTok}[1]{\textcolor[rgb]{0.81,0.36,0.00}{\textbf{#1}}}
\newcommand{\OtherTok}[1]{\textcolor[rgb]{0.56,0.35,0.01}{#1}}
\newcommand{\PreprocessorTok}[1]{\textcolor[rgb]{0.56,0.35,0.01}{\textit{#1}}}
\newcommand{\RegionMarkerTok}[1]{#1}
\newcommand{\SpecialCharTok}[1]{\textcolor[rgb]{0.81,0.36,0.00}{\textbf{#1}}}
\newcommand{\SpecialStringTok}[1]{\textcolor[rgb]{0.31,0.60,0.02}{#1}}
\newcommand{\StringTok}[1]{\textcolor[rgb]{0.31,0.60,0.02}{#1}}
\newcommand{\VariableTok}[1]{\textcolor[rgb]{0.00,0.00,0.00}{#1}}
\newcommand{\VerbatimStringTok}[1]{\textcolor[rgb]{0.31,0.60,0.02}{#1}}
\newcommand{\WarningTok}[1]{\textcolor[rgb]{0.56,0.35,0.01}{\textbf{\textit{#1}}}}
\usepackage{graphicx}
\makeatletter
\newsavebox\pandoc@box
\newcommand*\pandocbounded[1]{% scales image to fit in text height/width
  \sbox\pandoc@box{#1}%
  \Gscale@div\@tempa{\textheight}{\dimexpr\ht\pandoc@box+\dp\pandoc@box\relax}%
  \Gscale@div\@tempb{\linewidth}{\wd\pandoc@box}%
  \ifdim\@tempb\p@<\@tempa\p@\let\@tempa\@tempb\fi% select the smaller of both
  \ifdim\@tempa\p@<\p@\scalebox{\@tempa}{\usebox\pandoc@box}%
  \else\usebox{\pandoc@box}%
  \fi%
}
% Set default figure placement to htbp
\def\fps@figure{htbp}
\makeatother
\setlength{\emergencystretch}{3em} % prevent overfull lines
\providecommand{\tightlist}{%
  \setlength{\itemsep}{0pt}\setlength{\parskip}{0pt}}
\usepackage{fvextra}
\DefineVerbatimEnvironment{Highlighting}{Verbatim}{breaklines,commandchars=\\\{\}}
\usepackage{booktabs}
\usepackage{longtable}
\usepackage{array}
\usepackage{multirow}
\usepackage{wrapfig}
\usepackage{float}
\usepackage{colortbl}
\usepackage{pdflscape}
\usepackage{tabu}
\usepackage{threeparttable}
\usepackage{threeparttablex}
\usepackage[normalem]{ulem}
\usepackage{makecell}
\usepackage{xcolor}
\usepackage{bookmark}
\IfFileExists{xurl.sty}{\usepackage{xurl}}{} % add URL line breaks if available
\urlstyle{same}
\hypersetup{
  pdftitle={Analysis Report: Motivated Responses to a Masculinity Threat in a German Cultural Context},
  hidelinks,
  pdfcreator={LaTeX via pandoc}}

\title{Analysis Report: Motivated Responses to a Masculinity Threat in a
German Cultural Context}
\author{}
\date{\vspace{-2.5em}2025-12-02}

\begin{document}
\maketitle

\begin{Shaded}
\begin{Highlighting}[]
\NormalTok{df }\OtherTok{\textless{}{-}} \FunctionTok{read.csv}\NormalTok{(}\FunctionTok{here}\NormalTok{(}\StringTok{"data"}\NormalTok{, }\StringTok{"fragile\_masculinity\_motivation\_anonymized.csv"}\NormalTok{))}
\NormalTok{df }\OtherTok{\textless{}{-}} \FunctionTok{preprocessData}\NormalTok{(df, all\_items)}
\end{Highlighting}
\end{Shaded}

\begin{verbatim}
## Number of Observations: 196
## Condition threat: 101 participants
## Condition noThreat: 95 participants
\end{verbatim}

\section{1. Study Inclusion Criteria}\label{study-inclusion-criteria}

\subsection{1.1 Debrief Consent}\label{debrief-consent}

Participants that withdrew their consent after the debrief are excluded

\begin{Shaded}
\begin{Highlighting}[]
\NormalTok{df }\OtherTok{\textless{}{-}} \FunctionTok{exclude\_participants}\NormalTok{(}
\NormalTok{  df,}
\NormalTok{  DEBRIEFCONSENT }\SpecialCharTok{==} \StringTok{"Y"}\NormalTok{,}
  \AttributeTok{vars =} \StringTok{"DEBRIEFCONSENT"}\NormalTok{,}
  \AttributeTok{description =} \StringTok{"Debrief consent"}
\NormalTok{)}
\end{Highlighting}
\end{Shaded}

\begin{verbatim}
## Debrief consent: Excluded 8 participants (8 of 196).
## Remaining: 188
## Excluded responses summary for `DEBRIEFCONSENT`:  DEBRIEFCONSENT n
## 1              N 8
\end{verbatim}

\subsection{1.2 Demographic criteria}\label{demographic-criteria}

Participants have to be older than 18 years, self-identify as male and
native level German skills

\begin{Shaded}
\begin{Highlighting}[]
\NormalTok{df }\OtherTok{\textless{}{-}} \FunctionTok{exclude\_participants}\NormalTok{(df, AGE }\SpecialCharTok{\textgreater{}=} \DecValTok{18}\NormalTok{, }\AttributeTok{vars =} \StringTok{"AGE"}\NormalTok{, }\AttributeTok{description =} \StringTok{"Age"}\NormalTok{)}
\end{Highlighting}
\end{Shaded}

\begin{verbatim}
## Age: Excluded 3 participants (3 of 188).
## Remaining: 185
## Excluded responses summary for `AGE`:  AGE n
## 1   0 1
## 2   3 1
## 3  12 1
\end{verbatim}

\begin{Shaded}
\begin{Highlighting}[]
\NormalTok{df }\OtherTok{\textless{}{-}} \FunctionTok{exclude\_participants}\NormalTok{(df, SEX }\SpecialCharTok{==} \StringTok{"AO01"}\NormalTok{, }\AttributeTok{vars =} \StringTok{"SEX"}\NormalTok{, }\AttributeTok{description =} \StringTok{"Gender"}\NormalTok{)}
\end{Highlighting}
\end{Shaded}

\begin{verbatim}
## Gender: Excluded 9 participants (9 of 185).
## Remaining: 176
## Excluded responses summary for `SEX`:   SEX n
## 1 AO02 7
## 2 AO04 2
\end{verbatim}

\begin{Shaded}
\begin{Highlighting}[]
\NormalTok{df }\OtherTok{\textless{}{-}} \FunctionTok{exclude\_participants}\NormalTok{(df, GER }\SpecialCharTok{\%in\%} \FunctionTok{c}\NormalTok{(}\StringTok{"AO04"}\NormalTok{, }\StringTok{"AO05"}\NormalTok{), }\AttributeTok{vars =} \StringTok{"GER"}\NormalTok{, }\AttributeTok{description =} \StringTok{"German skills"}\NormalTok{)}
\end{Highlighting}
\end{Shaded}

\begin{verbatim}
## German skills: Excluded 1 participants (1 of 176).
## Remaining: 175
## Excluded responses summary for `GER`:   GER n
## 1 AO03 1
\end{verbatim}

\subsection{1.3 Suspicion}\label{suspicion}

Participants may not indicate a strong suspicion about the study (coded
in column SUSPICIONEXCLUSION, free text are excluded in anonymous data
set)

\begin{Shaded}
\begin{Highlighting}[]
\NormalTok{df }\OtherTok{\textless{}{-}} \FunctionTok{exclude\_participants}\NormalTok{(}
\NormalTok{  df,}
\NormalTok{  SUSPICIONEXCLUSION }\SpecialCharTok{==} \ConstantTok{FALSE}\NormalTok{,}
  \AttributeTok{vars =} \StringTok{"SUSPICIONEXCLUSION"}\NormalTok{,}
  \AttributeTok{description =} \StringTok{"Suspicion"}
\NormalTok{)}
\end{Highlighting}
\end{Shaded}

\begin{verbatim}
## Suspicion: Excluded 9 participants (9 of 175).
## Remaining: 166
## Excluded responses summary for `SUSPICIONEXCLUSION`:  SUSPICIONEXCLUSION n
## 1               TRUE 9
\end{verbatim}

\subsection{1.4 WFCT comprehension}\label{wfct-comprehension}

Participants should complete more than 50\% of word fragments with
existing words

\begin{Shaded}
\begin{Highlighting}[]
\NormalTok{df }\OtherTok{\textless{}{-}} \FunctionTok{exclude\_participants}\NormalTok{(}
\NormalTok{  df,}
\NormalTok{  validWordCompletionScore }\SpecialCharTok{\textgreater{}=} \FloatTok{0.5}\NormalTok{,}
  \AttributeTok{vars =} \StringTok{"validWordCompletionScore"}\NormalTok{,}
  \AttributeTok{description =} \StringTok{"WFCT comprehension"}
\NormalTok{)}
\end{Highlighting}
\end{Shaded}

\begin{verbatim}
## WFCT comprehension: Excluded 11 participants (11 of 166).
## Remaining: 155
## Excluded responses summary for `validWordCompletionScore`:  validWordCompletionScore n
## 1                     0.00 7
## 2                     0.40 2
## 3                     0.05 1
## 4                     0.15 1
\end{verbatim}

\subsection{1.5 Responses to Motivation for Masculine
Behaviour}\label{responses-to-motivation-for-masculine-behaviour}

Participants should respond to at least one item of each MMB scale

\begin{Shaded}
\begin{Highlighting}[]
\NormalTok{df }\OtherTok{\textless{}{-}} \FunctionTok{exclude\_participants}\NormalTok{(}
\NormalTok{  df,}
  \FunctionTok{rowSums}\NormalTok{(}\SpecialCharTok{!}\FunctionTok{is.na}\NormalTok{(df[mmb\_pressured\_items])) }\SpecialCharTok{\textgreater{}=} \DecValTok{1}\NormalTok{,}
  \AttributeTok{vars =}\NormalTok{ mmb\_pressured\_items,}
  \AttributeTok{description =} \StringTok{"Missing all Pressured items"}
\NormalTok{)}
\end{Highlighting}
\end{Shaded}

\begin{verbatim}
## Missing all Pressured items: Excluded 1 participants (1 of 155).
## Remaining: 154
## Excluded responses summary for `MMBi1`:  MMBi1 n
## 1    NA 1
## 
## Excluded responses summary for `MMBi2`:  MMBi2 n
## 1    NA 1
## 
## Excluded responses summary for `MMBi3`:  MMBi3 n
## 1    NA 1
## 
## Excluded responses summary for `MMBi4`:  MMBi4 n
## 1    NA 1
## 
## Excluded responses summary for `MMBi5`:  MMBi5 n
## 1    NA 1
\end{verbatim}

\begin{Shaded}
\begin{Highlighting}[]
\NormalTok{df }\OtherTok{\textless{}{-}} \FunctionTok{exclude\_participants}\NormalTok{(}
\NormalTok{  df,}
  \FunctionTok{rowSums}\NormalTok{(}\SpecialCharTok{!}\FunctionTok{is.na}\NormalTok{(df[mmb\_autonomous\_items])) }\SpecialCharTok{\textgreater{}=} \DecValTok{1}\NormalTok{,}
  \AttributeTok{vars =}\NormalTok{ mmb\_autonomous\_items,}
  \AttributeTok{description =} \StringTok{"Missing all Autonomous items"}
\NormalTok{)}
\end{Highlighting}
\end{Shaded}

\begin{verbatim}
## Missing all Autonomous items: Excluded 0 participants (0 of 154).
## Remaining: 154
\end{verbatim}

\section{2. Demographics}\label{demographics}

\subsection{2.2 Age}\label{age}

\begin{Shaded}
\begin{Highlighting}[]
\FunctionTok{summary\_mean\_sd}\NormalTok{(df, }\StringTok{"AGE"}\NormalTok{)}
\end{Highlighting}
\end{Shaded}

\begin{verbatim}
## # A tibble: 1 x 5
##   variable  mean    sd     n missing
##   <chr>    <dbl> <dbl> <dbl>   <dbl>
## 1 AGE       30.6  10.9   154       0
\end{verbatim}

\begin{Shaded}
\begin{Highlighting}[]
\FunctionTok{plotHist}\NormalTok{(}\StringTok{"AGE"}\NormalTok{, }\FunctionTok{c}\NormalTok{(}\DecValTok{0}\NormalTok{, }\DecValTok{70}\NormalTok{), }\AttributeTok{condition =} \ConstantTok{NULL}\NormalTok{, }\AttributeTok{binwidth =} \DecValTok{1}\NormalTok{)}
\end{Highlighting}
\end{Shaded}

\begin{center}\includegraphics{analysis-report_files/figure-latex/unnamed-chunk-10-1} \end{center}

\subsection{2.3 Educational Degree}\label{educational-degree}

\begin{Shaded}
\begin{Highlighting}[]
\FunctionTok{freq\_table}\NormalTok{(df, EDU\_label)}
\end{Highlighting}
\end{Shaded}

\begin{verbatim}
## # A tibble: 5 x 3
##   EDU_label                                                        Count Percent
##   <chr>                                                            <int>   <dbl>
## 1 "Abitur, allgemeine\noder fachgebundene\nHochschulreife / (EOS)"   124    80.5
## 2 "Fachhochschulreife /\nFachoberschule"                              22    14.3
## 3 "Mittlerer\nSchulabschluss /\nRealschule / mittlere\nReife"          6     3.9
## 4 "Haupt-/ Volksschule"                                                1     0.6
## 5 "Noch kein\nAbschluss/Schüler"                                       1     0.6
\end{verbatim}

\begin{Shaded}
\begin{Highlighting}[]
\FunctionTok{plotBarHorizontal}\NormalTok{(}\StringTok{"EDU\_label"}\NormalTok{) }\SpecialCharTok{+}
  \FunctionTok{labs}\NormalTok{(}
    \AttributeTok{x =} \ConstantTok{NULL}\NormalTok{,}
    \AttributeTok{y =} \StringTok{"Anzahl"}
\NormalTok{  )}
\end{Highlighting}
\end{Shaded}

\begin{center}\includegraphics{analysis-report_files/figure-latex/unnamed-chunk-12-1} \end{center}

\begin{Shaded}
\begin{Highlighting}[]
\FunctionTok{freq\_table}\NormalTok{(df, EDUPUPIL\_label)}
\end{Highlighting}
\end{Shaded}

\begin{verbatim}
## # A tibble: 2 x 3
##   EDUPUPIL_label      Count Percent
##   <chr>               <int>   <dbl>
## 1 <NA>                  153    99.4
## 2 Haupt-/ Volksschule     1     0.6
\end{verbatim}

\subsection{2.4 Occupational Degree}\label{occupational-degree}

\begin{Shaded}
\begin{Highlighting}[]
\FunctionTok{freq\_table}\NormalTok{(df, OCC\_label)}
\end{Highlighting}
\end{Shaded}

\begin{verbatim}
## # A tibble: 6 x 3
##   OCC_label                                                        Count Percent
##   <chr>                                                            <int>   <dbl>
## 1 "Universität / Hochschule"                                          93    60.4
## 2 "Noch keinen\nBerufsabschluss"                                      32    20.8
## 3 "Ausbildung an\nBerufsfachschule,\nHandelsschule (beruflich-\ns~    13     8.4
## 4 "Fachschule (Meister-,\nTechnikerschule, Berufs-\noder Fachakad~     7     4.5
## 5 "Lehre\n(beruflich-betrieblich)"                                     5     3.2
## 6 "Fachhochschule /\nIngenieurschule"                                  4     2.6
\end{verbatim}

\begin{Shaded}
\begin{Highlighting}[]
\FunctionTok{plotBarHorizontal}\NormalTok{(}\StringTok{"OCC\_label"}\NormalTok{) }\SpecialCharTok{+}
  \FunctionTok{labs}\NormalTok{(}
    \AttributeTok{x =} \ConstantTok{NULL}\NormalTok{,}
    \AttributeTok{y =} \StringTok{"Anzahl"}
\NormalTok{  )}
\end{Highlighting}
\end{Shaded}

\begin{center}\includegraphics{analysis-report_files/figure-latex/unnamed-chunk-15-1} \end{center}

\section{3. Experimental Manipulation
Check}\label{experimental-manipulation-check}

\subsection{3.1 Conditions}\label{conditions}

\begin{Shaded}
\begin{Highlighting}[]
\FunctionTok{printNumberOfParticipants}\NormalTok{(df, }\StringTok{"threatCondition"}\NormalTok{)}
\end{Highlighting}
\end{Shaded}

\begin{verbatim}
## Number of Observations: 154
## Condition threat: 75 participants
## Condition noThreat: 79 participants
\end{verbatim}

\subsection{3.2 Gender Knowledge
Feedback}\label{gender-knowledge-feedback}

{[}GKFEEDBACK{]} Welche Rückmeldung haben Sie im Verlauf der Studie zu
Ihrem Wissen in geschlechtsspezifischen Themen erhalten? Scale 1 to 10

\begin{Shaded}
\begin{Highlighting}[]
\NormalTok{summary\_gkfeedback }\OtherTok{\textless{}{-}} \FunctionTok{summary\_mean\_sd}\NormalTok{(df, }\StringTok{"GKFEEDBACK"}\NormalTok{, }\StringTok{"threatCondition"}\NormalTok{)}
\NormalTok{summary\_gkfeedback}
\end{Highlighting}
\end{Shaded}

\begin{verbatim}
## # A tibble: 2 x 6
##   threatCondition variable    mean    sd     n missing
##   <fct>           <chr>      <dbl> <dbl> <dbl>   <dbl>
## 1 threat          GKFEEDBACK  3.08  0.93    75       0
## 2 noThreat        GKFEEDBACK  6.22  1       79       0
\end{verbatim}

\begin{Shaded}
\begin{Highlighting}[]
\FunctionTok{plotHist}\NormalTok{(}\StringTok{"GKFEEDBACK"}\NormalTok{, }\FunctionTok{c}\NormalTok{(}\DecValTok{1}\NormalTok{, }\DecValTok{10}\NormalTok{), }\StringTok{"threatCondition"}\NormalTok{)}
\end{Highlighting}
\end{Shaded}

\begin{center}\includegraphics{analysis-report_files/figure-latex/unnamed-chunk-18-1} \end{center}

\textbf{Interpretation}

\begin{itemize}
\tightlist
\item
  There is a clar split between the two conditions, meaning that
  participants paid attention to the feedback on the GenderKnoledge task
  and that the feedback was comprehensible
\end{itemize}

\subsection{3.3 Self evaluation of Gender
Knowledge}\label{self-evaluation-of-gender-knowledge}

{[}GKSELF{]} Wie würden Sie Ihr Wissen in geschlechtsspezifischen Themen
einschätzen?

\begin{Shaded}
\begin{Highlighting}[]
\NormalTok{summary\_gkself }\OtherTok{\textless{}{-}} \FunctionTok{summary\_mean\_sd}\NormalTok{(df, }\StringTok{"GKSELF"}\NormalTok{, }\StringTok{"threatCondition"}\NormalTok{)}
\NormalTok{summary\_gkself}
\end{Highlighting}
\end{Shaded}

\begin{verbatim}
## # A tibble: 2 x 6
##   threatCondition variable  mean    sd     n missing
##   <fct>           <chr>    <dbl> <dbl> <dbl>   <dbl>
## 1 threat          GKSELF    5.71  1.43    73       2
## 2 noThreat        GKSELF    6.79  1.38    78       1
\end{verbatim}

\begin{Shaded}
\begin{Highlighting}[]
\FunctionTok{plotHist}\NormalTok{(}\StringTok{"GKSELF"}\NormalTok{, }\FunctionTok{c}\NormalTok{(}\DecValTok{1}\NormalTok{, }\DecValTok{10}\NormalTok{), }\StringTok{"threatCondition"}\NormalTok{)}
\end{Highlighting}
\end{Shaded}

\begin{center}\includegraphics{analysis-report_files/figure-latex/unnamed-chunk-20-1} \end{center}

\textbf{Interpretation}

\begin{itemize}
\tightlist
\item
  The mean of self evalutated gender knowledge is lower in the threat
  condition, than noThreat condition
\item
  This shows that the feedback did have an effect on the self perception
  of gender knowledge
\item
  The split is not very clear though\ldots{}
\end{itemize}

\subsection{3.4 Valid Word Completions}\label{valid-word-completions}

\begin{Shaded}
\begin{Highlighting}[]
\FunctionTok{summary\_mean\_sd}\NormalTok{(df, }\StringTok{"validWordCompletionScore"}\NormalTok{)}
\end{Highlighting}
\end{Shaded}

\begin{verbatim}
## # A tibble: 1 x 5
##   variable                  mean    sd     n missing
##   <chr>                    <dbl> <dbl> <dbl>   <dbl>
## 1 validWordCompletionScore  0.92   0.1   154       0
\end{verbatim}

\begin{Shaded}
\begin{Highlighting}[]
\FunctionTok{plotBox}\NormalTok{(}\StringTok{"validWordCompletionScore"}\NormalTok{)}
\end{Highlighting}
\end{Shaded}

\begin{center}\includegraphics{analysis-report_files/figure-latex/unnamed-chunk-22-1} \end{center}

\textbf{Interpretation}

\begin{itemize}
\tightlist
\item
  Participants with a score lower than 50\% were excluded
\item
  Generally participants were able to complete the WCFT
\end{itemize}

\subsection{3.5 Motivation for Masculine
Behaviour}\label{motivation-for-masculine-behaviour}

\begin{Shaded}
\begin{Highlighting}[]
\NormalTok{plots }\OtherTok{\textless{}{-}} \FunctionTok{lapply}\NormalTok{(MMB\_item\_codes, }\ControlFlowTok{function}\NormalTok{(item) \{}
  \FunctionTok{plotHist}\NormalTok{(item, }\FunctionTok{c}\NormalTok{(}\DecValTok{1}\NormalTok{, }\DecValTok{7}\NormalTok{), }\AttributeTok{condition =} \ConstantTok{NULL}\NormalTok{, }\AttributeTok{binwidth =} \DecValTok{1}\NormalTok{)}
\NormalTok{\})}

\FunctionTok{do.call}\NormalTok{(grid.arrange, }\FunctionTok{c}\NormalTok{(plots, }\AttributeTok{ncol =} \DecValTok{3}\NormalTok{))}
\end{Highlighting}
\end{Shaded}

\begin{center}\includegraphics{analysis-report_files/figure-latex/unnamed-chunk-23-1} \end{center}

\begin{Shaded}
\begin{Highlighting}[]
\FunctionTok{summary\_mean\_sd}\NormalTok{(df, MMB\_item\_codes)}
\end{Highlighting}
\end{Shaded}

\begin{verbatim}
## # A tibble: 9 x 5
##   variable  mean    sd     n missing
##   <chr>    <dbl> <dbl> <dbl>   <dbl>
## 1 MMBi1     3.55  1.78   154       0
## 2 MMBi2     3.75  1.74   154       0
## 3 MMBi3     3.35  1.68   154       0
## 4 MMBi4     3.51  1.74   154       0
## 5 MMBi5     2.97  1.68   154       0
## 6 MMBi6     4.06  1.79   154       0
## 7 MMBi7     5.29  1.49   154       0
## 8 MMBi8     4.32  1.68   154       0
## 9 MMBi9     3.26  1.77   154       0
\end{verbatim}

\subsection{3.6 Aggressive Cognition}\label{aggressive-cognition}

\begin{Shaded}
\begin{Highlighting}[]
\NormalTok{plots }\OtherTok{\textless{}{-}} \FunctionTok{lapply}\NormalTok{(WFCT\_aggression\_items, }\ControlFlowTok{function}\NormalTok{(item) \{}
  \FunctionTok{plotHist}\NormalTok{(item, }\FunctionTok{c}\NormalTok{(}\DecValTok{0}\NormalTok{, }\DecValTok{1}\NormalTok{), }\AttributeTok{condition =} \ConstantTok{NULL}\NormalTok{, }\AttributeTok{binwidth =} \DecValTok{1}\NormalTok{)}
\NormalTok{\})}

\FunctionTok{do.call}\NormalTok{(grid.arrange, }\FunctionTok{c}\NormalTok{(plots, }\AttributeTok{ncol =} \DecValTok{4}\NormalTok{))}
\end{Highlighting}
\end{Shaded}

\begin{center}\includegraphics{analysis-report_files/figure-latex/unnamed-chunk-25-1} \end{center}

\begin{Shaded}
\begin{Highlighting}[]
\FunctionTok{summary\_mean\_sd}\NormalTok{(df, WFCT\_aggression\_items)}
\end{Highlighting}
\end{Shaded}

\begin{verbatim}
## # A tibble: 10 x 5
##    variable   mean    sd     n missing
##    <chr>     <dbl> <dbl> <dbl>   <dbl>
##  1 WFCTagg1   0.05  0.21   146       8
##  2 WFCTagg2   0.13  0.34   150       4
##  3 WFCTagg3   0.07  0.26   149       5
##  4 WFCTagg4   0.08  0.27   148       6
##  5 WFCTagg5   0.03  0.18   151       3
##  6 WFCTagg6   0.51  0.5    152       2
##  7 WFCTagg7   0.1   0.3    143      11
##  8 WFCTagg8   0.41  0.49   140      14
##  9 WFCTagg9   0.23  0.42   142      12
## 10 WFCTagg10  0.64  0.48   141      13
\end{verbatim}

\subsection{3.7 Anxious Cognition}\label{anxious-cognition}

\begin{Shaded}
\begin{Highlighting}[]
\NormalTok{plots }\OtherTok{\textless{}{-}} \FunctionTok{lapply}\NormalTok{(WFCT\_anxiety\_items, }\ControlFlowTok{function}\NormalTok{(item) \{}
  \FunctionTok{plotHist}\NormalTok{(item, }\FunctionTok{c}\NormalTok{(}\DecValTok{0}\NormalTok{, }\DecValTok{1}\NormalTok{), }\AttributeTok{condition =} \ConstantTok{NULL}\NormalTok{, }\AttributeTok{binwidth =} \DecValTok{1}\NormalTok{)}
\NormalTok{\})}

\FunctionTok{do.call}\NormalTok{(grid.arrange, }\FunctionTok{c}\NormalTok{(plots, }\AttributeTok{ncol =} \DecValTok{4}\NormalTok{))}
\end{Highlighting}
\end{Shaded}

\begin{center}\includegraphics{analysis-report_files/figure-latex/unnamed-chunk-27-1} \end{center}

\begin{Shaded}
\begin{Highlighting}[]
\FunctionTok{summary\_mean\_sd}\NormalTok{(df, WFCT\_anxiety\_items)}
\end{Highlighting}
\end{Shaded}

\begin{verbatim}
## # A tibble: 10 x 5
##    variable   mean    sd     n missing
##    <chr>     <dbl> <dbl> <dbl>   <dbl>
##  1 WFCTanx1   0.31  0.47   144      10
##  2 WFCTanx2   0.03  0.18   145       9
##  3 WFCTanx3   0.85  0.36   108      46
##  4 WFCTanx4   0.26  0.44   145       9
##  5 WFCTanx5   0.33  0.47   140      14
##  6 WFCTanx6   0.36  0.48   148       6
##  7 WFCTanx7   0.16  0.37   144      10
##  8 WFCTanx8   0.08  0.27   151       3
##  9 WFCTanx9   0.86  0.35   105      49
## 10 WFCTanx10  0.17  0.37   144      10
\end{verbatim}

\section{4. Test Quality Criteria: Motivation for Masculine
Behaviour}\label{test-quality-criteria-motivation-for-masculine-behaviour}

\emph{Objectives:}

\begin{itemize}
\tightlist
\item
  Assess the \emph{validity} and \emph{reliability} of the scale
  Motivation for Masculine Behaviour (MMB)
\item
  Compute participant latent trait scores
\end{itemize}

\emph{Methodology:}

\begin{itemize}
\tightlist
\item
  Visualization \& Descriptives
\item
  CFA and EFA to =\textgreater{} construct validity
\item
  McDonald's Omega =\textgreater{} Reliabiltiy
\item
  Regression scoring =\textgreater{} latent trait scores
\end{itemize}

\subsection{4.1 Construct Validity}\label{construct-validity}

\subsubsection{4.1.1 Visualization \&
Descriptives}\label{visualization-descriptives}

\begin{Shaded}
\begin{Highlighting}[]
\NormalTok{cor\_mmb }\OtherTok{\textless{}{-}}\NormalTok{ psych}\SpecialCharTok{::}\FunctionTok{polychoric}\NormalTok{(df[MMB\_item\_codes])}\SpecialCharTok{$}\NormalTok{rho}

\FunctionTok{corrplot}\NormalTok{(cor\_mmb,}
  \AttributeTok{method =} \StringTok{"color"}\NormalTok{,}
  \AttributeTok{type =} \StringTok{"upper"}\NormalTok{,}
  \AttributeTok{order =} \StringTok{"original"}\NormalTok{,}
  \AttributeTok{tl.col =} \StringTok{"black"}\NormalTok{,}
  \AttributeTok{tl.srt =} \DecValTok{45}\NormalTok{,}
  \AttributeTok{diag =} \ConstantTok{FALSE}
\NormalTok{)}
\end{Highlighting}
\end{Shaded}

\begin{center}\includegraphics{analysis-report_files/figure-latex/unnamed-chunk-29-1} \end{center}

\textbf{Interpretation}

\begin{itemize}
\tightlist
\item
  The correlation plit shows a very clear block for MMB 1-5 items
  (pressured scale)
\item
  The block for for autonomous items (MMB 6 - 9) is way less clear
\end{itemize}

\begin{Shaded}
\begin{Highlighting}[]
\FunctionTok{summary\_mean\_sd}\NormalTok{(df, MMB\_item\_codes)}
\end{Highlighting}
\end{Shaded}

\begin{verbatim}
## # A tibble: 9 x 5
##   variable  mean    sd     n missing
##   <chr>    <dbl> <dbl> <dbl>   <dbl>
## 1 MMBi1     3.55  1.78   154       0
## 2 MMBi2     3.75  1.74   154       0
## 3 MMBi3     3.35  1.68   154       0
## 4 MMBi4     3.51  1.74   154       0
## 5 MMBi5     2.97  1.68   154       0
## 6 MMBi6     4.06  1.79   154       0
## 7 MMBi7     5.29  1.49   154       0
## 8 MMBi8     4.32  1.68   154       0
## 9 MMBi9     3.26  1.77   154       0
\end{verbatim}

\textbf{Interpretation}

\begin{itemize}
\tightlist
\item
  The item scores ranged between 1 and 7
\item
  Most items show a mean in the center of the scale
\item
  Item MMBi7 shows a very high mean (5.27). ``Ich bin gerne männlich''\\
\item
  Generally, the items do not show bottom or ceiling effects, and there
  is some variance to the responses
\end{itemize}

\subsubsection{4.1.2 Check sampling adequacy for Factory
Analysis}\label{check-sampling-adequacy-for-factory-analysis}

\begin{Shaded}
\begin{Highlighting}[]
\FunctionTok{KMO}\NormalTok{(cor\_mmb)}
\end{Highlighting}
\end{Shaded}

\begin{verbatim}
## Kaiser-Meyer-Olkin factor adequacy
## Call: KMO(r = cor_mmb)
## Overall MSA =  0.85
## MSA for each item = 
## MMBi1 MMBi2 MMBi3 MMBi4 MMBi5 MMBi6 MMBi7 MMBi8 MMBi9 
##  0.92  0.88  0.89  0.88  0.83  0.83  0.69  0.81  0.80
\end{verbatim}

\begin{Shaded}
\begin{Highlighting}[]
\FunctionTok{cortest.bartlett}\NormalTok{(cor\_mmb, }\AttributeTok{n =} \FunctionTok{nrow}\NormalTok{(df))}
\end{Highlighting}
\end{Shaded}

\begin{verbatim}
## $chisq
## [1] 1034.431
## 
## $p.value
## [1] 9.376131e-194
## 
## $df
## [1] 36
\end{verbatim}

KMO (Kaiser-Meyer-Olkin) should be \textgreater{} 0.6 (ideally
\textgreater{} 0.8) Bartlett's test should be significant (p \textless{}
.05)

\textbf{Interpretation}

\begin{itemize}
\tightlist
\item
  The overall KMO is 0.86 showing a good smapling adequacy for factor
  analysis
\item
  Individual scores of items are good as well, except MMBi7, which has a
  sufficient, yet lower KMO value (0.72) due to it's ceiling effect
\item
  The Bartlett test is significant as well, showing a sampling adequacy
  for factor analysis
\end{itemize}

\subsubsection{4.1.3 Two-Factor Uncorrelated
CFA}\label{two-factor-uncorrelated-cfa}

Testing the proposed 2 factor structure by Stanaland \& Gaither (2021)

\begin{itemize}
\tightlist
\item
  2 factors
\item
  no covariance between factors
\end{itemize}

\begin{Shaded}
\begin{Highlighting}[]
\NormalTok{model\_twof }\OtherTok{\textless{}{-}} \StringTok{"}
\StringTok{  \# latent variables}
\StringTok{  Pressured =\textasciitilde{} MMBi1 + MMBi2 + MMBi3 + MMBi4 + MMBi5}
\StringTok{  Autonomous =\textasciitilde{} MMBi6 + MMBi7 + MMBi8 + MMBi9}

\StringTok{  \# constrain covariance to 0}
\StringTok{  Pressured \textasciitilde{}\textasciitilde{} 0*Autonomous}
\StringTok{"}

\NormalTok{results\_twof }\OtherTok{\textless{}{-}} \FunctionTok{run\_cfa\_model}\NormalTok{(model\_twof, df, }\AttributeTok{model\_name =} \StringTok{"Two{-}Factor Uncorrelated"}\NormalTok{, }\AttributeTok{show\_mi =} \ConstantTok{TRUE}\NormalTok{)}
\end{Highlighting}
\end{Shaded}

\begin{verbatim}
## 
## -------------------------
## MODEL SUMMARY
## -------------------------
## lavaan 0.6.15 ended normally after 22 iterations
## 
##   Estimator                                         ML
##   Optimization method                           NLMINB
##   Number of model parameters                        18
## 
##   Number of observations                           154
## 
## Model Test User Model:
##                                                       
##   Test statistic                               159.497
##   Degrees of freedom                                27
##   P-value (Chi-square)                           0.000
## 
## Model Test Baseline Model:
## 
##   Test statistic                               923.528
##   Degrees of freedom                                36
##   P-value                                        0.000
## 
## User Model versus Baseline Model:
## 
##   Comparative Fit Index (CFI)                    0.851
##   Tucker-Lewis Index (TLI)                       0.801
## 
## Loglikelihood and Information Criteria:
## 
##   Loglikelihood user model (H0)              -2317.331
##   Loglikelihood unrestricted model (H1)      -2237.582
##                                                       
##   Akaike (AIC)                                4670.661
##   Bayesian (BIC)                              4725.326
##   Sample-size adjusted Bayesian (SABIC)       4668.354
## 
## Root Mean Square Error of Approximation:
## 
##   RMSEA                                          0.179
##   90 Percent confidence interval - lower         0.152
##   90 Percent confidence interval - upper         0.206
##   P-value H_0: RMSEA <= 0.050                    0.000
##   P-value H_0: RMSEA >= 0.080                    1.000
## 
## Standardized Root Mean Square Residual:
## 
##   SRMR                                           0.248
## 
## Parameter Estimates:
## 
##   Standard errors                             Standard
##   Information                                 Expected
##   Information saturated (h1) model          Structured
## 
## Latent Variables:
##                    Estimate  Std.Err  z-value  P(>|z|)   Std.lv  Std.all
##   Pressured =~                                                          
##     MMBi1             1.529    0.116   13.203    0.000    1.529    0.862
##     MMBi2             1.491    0.114   13.101    0.000    1.491    0.858
##     MMBi3             1.439    0.110   13.104    0.000    1.439    0.858
##     MMBi4             1.553    0.111   14.012    0.000    1.553    0.894
##     MMBi5             1.246    0.117   10.617    0.000    1.246    0.746
##   Autonomous =~                                                         
##     MMBi6             1.458    0.134   10.848    0.000    1.458    0.818
##     MMBi7             0.893    0.119    7.523    0.000    0.893    0.603
##     MMBi8             1.304    0.127   10.254    0.000    1.304    0.781
##     MMBi9             1.064    0.142    7.520    0.000    1.064    0.603
## 
## Covariances:
##                    Estimate  Std.Err  z-value  P(>|z|)   Std.lv  Std.all
##   Pressured ~~                                                          
##     Autonomous        0.000                               0.000    0.000
## 
## Variances:
##                    Estimate  Std.Err  z-value  P(>|z|)   Std.lv  Std.all
##    .MMBi1             0.807    0.118    6.857    0.000    0.807    0.257
##    .MMBi2             0.797    0.115    6.929    0.000    0.797    0.264
##    .MMBi3             0.741    0.107    6.927    0.000    0.741    0.264
##    .MMBi4             0.606    0.099    6.131    0.000    0.606    0.201
##    .MMBi5             1.239    0.156    7.961    0.000    1.239    0.444
##    .MMBi6             1.048    0.226    4.629    0.000    1.048    0.330
##    .MMBi7             1.394    0.182    7.670    0.000    1.394    0.636
##    .MMBi8             1.088    0.199    5.457    0.000    1.088    0.390
##    .MMBi9             1.982    0.258    7.671    0.000    1.982    0.636
##     Pressured         1.000                               1.000    1.000
##     Autonomous        1.000                               1.000    1.000
## 
## 
## -------------------------
## MODEL FIT INDICES
## -------------------------
##       Index Value Criteria
## cfi     cfi 0.851     0.95
## tli     tli 0.801     0.95
## rmsea rmsea 0.179     0.06
## srmr   srmr 0.248     0.08
## 
## -------------------------
## STANDARDIZED FACTOR LOADINGS
## -------------------------
##       Factor  Item Std_Loading
## 1  Pressured MMBi1       0.862
## 2  Pressured MMBi2       0.858
## 3  Pressured MMBi3       0.858
## 4  Pressured MMBi4       0.894
## 5  Pressured MMBi5       0.746
## 6 Autonomous MMBi6       0.818
## 7 Autonomous MMBi7       0.603
## 8 Autonomous MMBi8       0.781
## 9 Autonomous MMBi9       0.603
## 
## -------------------------
## TOP MODIFICATION INDICES
## -------------------------
##           lhs op        rhs     mi   epc sepc.lv sepc.all sepc.nox
## 10  Pressured ~~ Autonomous 37.076 0.559   0.559    0.559    0.559
## 60      MMBi5 ~~      MMBi9 30.371 0.767   0.767    0.489    0.489
## 64      MMBi7 ~~      MMBi8 19.500 0.745   0.745    0.605    0.605
## 63      MMBi6 ~~      MMBi9 19.500 0.992   0.992    0.689    0.689
## 28 Autonomous =~      MMBi3 17.903 0.359   0.359    0.214    0.214
\end{verbatim}

\textbf{Interpretation}

\begin{itemize}
\tightlist
\item
  None of the model fit indices is satisfying
\item
  Mod Indices suggest a correlation between latent pressured and
  autonomous factors
\end{itemize}

\subsubsection{4.1.4 Exploratory Factor
Analysis}\label{exploratory-factor-analysis}

\paragraph{Determine the number of latent
factors}\label{determine-the-number-of-latent-factors}

\begin{Shaded}
\begin{Highlighting}[]
\FunctionTok{fa.parallel}\NormalTok{(cor\_mmb,}
  \AttributeTok{fm =} \StringTok{"ml"}\NormalTok{,}
  \AttributeTok{fa =} \StringTok{"fa"}\NormalTok{, }\AttributeTok{n.iter =} \DecValTok{100}\NormalTok{, }\AttributeTok{show.legend =} \ConstantTok{TRUE}\NormalTok{,}
  \AttributeTok{main =} \StringTok{"Parallel Analysis Scree Plot"}
\NormalTok{)}
\end{Highlighting}
\end{Shaded}

\begin{center}\includegraphics{analysis-report_files/figure-latex/unnamed-chunk-33-1} \end{center}

\begin{verbatim}
## Parallel analysis suggests that the number of factors =  2  and the number of components =  NA
\end{verbatim}

\textbf{Interpretation:}

\begin{itemize}
\tightlist
\item
  The ``elbow'' of the scree plot suggest a two factor model
\end{itemize}

\paragraph{Compute EFA}\label{compute-efa}

\begin{Shaded}
\begin{Highlighting}[]
\NormalTok{efa\_result\_twof }\OtherTok{\textless{}{-}} \FunctionTok{fa}\NormalTok{(df[MMB\_item\_codes], }\AttributeTok{nfactors =} \DecValTok{2}\NormalTok{, }\AttributeTok{rotate =} \StringTok{"oblimin"}\NormalTok{, }\AttributeTok{fm =} \StringTok{"ml"}\NormalTok{)}
\FunctionTok{print}\NormalTok{(efa\_result\_twof, }\AttributeTok{cut =} \FloatTok{0.3}\NormalTok{) }\CommentTok{\# only loadings \textgreater{} .30}
\end{Highlighting}
\end{Shaded}

\begin{verbatim}
## Factor Analysis using method =  ml
## Call: fa(r = df[MMB_item_codes], nfactors = 2, rotate = "oblimin", 
##     fm = "ml")
## Standardized loadings (pattern matrix) based upon correlation matrix
##         ML1   ML2   h2   u2 com
## MMBi1  0.83       0.75 0.25 1.0
## MMBi2  0.90       0.75 0.25 1.0
## MMBi3  0.79       0.78 0.22 1.1
## MMBi4  0.89       0.79 0.21 1.0
## MMBi5  0.78       0.58 0.42 1.0
## MMBi6        0.69 0.64 0.36 1.2
## MMBi7        0.74 0.47 0.53 1.2
## MMBi8        0.80 0.67 0.33 1.0
## MMBi9  0.39  0.39 0.42 0.58 2.0
## 
##                        ML1  ML2
## SS loadings           3.87 1.96
## Proportion Var        0.43 0.22
## Cumulative Var        0.43 0.65
## Proportion Explained  0.66 0.34
## Cumulative Proportion 0.66 1.00
## 
##  With factor correlations of 
##     ML1 ML2
## ML1 1.0 0.4
## ML2 0.4 1.0
## 
## Mean item complexity =  1.2
## Test of the hypothesis that 2 factors are sufficient.
## 
## df null model =  36  with the objective function =  6 with Chi Square =  894.54
## df of  the model are 19  and the objective function was  0.38 
## 
## The root mean square of the residuals (RMSR) is  0.04 
## The df corrected root mean square of the residuals is  0.06 
## 
## The harmonic n.obs is  154 with the empirical chi square  20.92  with prob <  0.34 
## The total n.obs was  154  with Likelihood Chi Square =  56.32  with prob <  1.5e-05 
## 
## Tucker Lewis Index of factoring reliability =  0.917
## RMSEA index =  0.113  and the 90 % confidence intervals are  0.08 0.148
## BIC =  -39.38
## Fit based upon off diagonal values = 0.99
## Measures of factor score adequacy             
##                                                    ML1  ML2
## Correlation of (regression) scores with factors   0.97 0.91
## Multiple R square of scores with factors          0.94 0.83
## Minimum correlation of possible factor scores     0.87 0.66
\end{verbatim}

\begin{Shaded}
\begin{Highlighting}[]
\CommentTok{\# Create loadings matrix}
\NormalTok{loadings\_matrix }\OtherTok{\textless{}{-}} \FunctionTok{as.data.frame}\NormalTok{(efa\_result\_twof}\SpecialCharTok{$}\NormalTok{loadings[}\DecValTok{1}\SpecialCharTok{:}\FunctionTok{length}\NormalTok{(mmb\_labels), ])}

\CommentTok{\# Format table}
\NormalTok{loading\_table }\OtherTok{\textless{}{-}} \FunctionTok{data.frame}\NormalTok{(}
  \AttributeTok{Item =}\NormalTok{ MMB\_item\_codes,}
  \AttributeTok{Label =}\NormalTok{ mmb\_labels,}
  \AttributeTok{Factor1 =} \FunctionTok{round}\NormalTok{(loadings\_matrix}\SpecialCharTok{$}\NormalTok{ML1, }\DecValTok{2}\NormalTok{),}
  \AttributeTok{Factor2 =} \FunctionTok{round}\NormalTok{(loadings\_matrix}\SpecialCharTok{$}\NormalTok{ML2, }\DecValTok{2}\NormalTok{)}
\NormalTok{)}

\CommentTok{\# Hide loadings \textless{} .30}
\NormalTok{loading\_table }\OtherTok{\textless{}{-}}\NormalTok{ loading\_table }\SpecialCharTok{\%\textgreater{}\%}
  \FunctionTok{mutate}\NormalTok{(}
    \AttributeTok{Factor1 =} \FunctionTok{ifelse}\NormalTok{(}\FunctionTok{abs}\NormalTok{(Factor1) }\SpecialCharTok{\textless{}} \FloatTok{0.30}\NormalTok{, }\StringTok{""}\NormalTok{, }\FunctionTok{as.character}\NormalTok{(Factor1)),}
    \AttributeTok{Factor2 =} \FunctionTok{ifelse}\NormalTok{(}\FunctionTok{abs}\NormalTok{(Factor2) }\SpecialCharTok{\textless{}} \FloatTok{0.30}\NormalTok{, }\StringTok{""}\NormalTok{, }\FunctionTok{as.character}\NormalTok{(Factor2))}
\NormalTok{  )}

\CommentTok{\# Render nice APA{-}style table}
\NormalTok{loading\_table }\SpecialCharTok{\%\textgreater{}\%}
  \FunctionTok{kbl}\NormalTok{(}
    \AttributeTok{caption =} \StringTok{"EFA Loadings for MMB Items (2{-}Factor Solution)"}\NormalTok{,}
    \AttributeTok{align =} \StringTok{"lccc"}\NormalTok{,}
    \AttributeTok{booktabs =} \ConstantTok{TRUE}
\NormalTok{  ) }\SpecialCharTok{\%\textgreater{}\%}
  \FunctionTok{kable\_styling}\NormalTok{(}
    \AttributeTok{full\_width =} \ConstantTok{FALSE}\NormalTok{,}
    \AttributeTok{position =} \StringTok{"center"}\NormalTok{,}
    \AttributeTok{bootstrap\_options =} \FunctionTok{c}\NormalTok{(}\StringTok{"striped"}\NormalTok{, }\StringTok{"hover"}\NormalTok{, }\StringTok{"condensed"}\NormalTok{),}
    \AttributeTok{latex\_options =} \StringTok{"hold\_position"}
\NormalTok{  ) }\SpecialCharTok{\%\textgreater{}\%}
  \FunctionTok{column\_spec}\NormalTok{(}\DecValTok{2}\NormalTok{, }\AttributeTok{width =} \StringTok{"10cm"}\NormalTok{) }\SpecialCharTok{\%\textgreater{}\%}  
  \FunctionTok{row\_spec}\NormalTok{(}\DecValTok{0}\NormalTok{, }\AttributeTok{bold =} \ConstantTok{TRUE}\NormalTok{, }\AttributeTok{background =} \StringTok{"\#f2f2f2"}\NormalTok{) }\SpecialCharTok{\%\textgreater{}\%}
  \FunctionTok{row\_spec}\NormalTok{(}\DecValTok{1}\SpecialCharTok{:}\FunctionTok{nrow}\NormalTok{(loading\_table), }\AttributeTok{background =} \StringTok{"white"}\NormalTok{, }\AttributeTok{color =} \StringTok{"black"}\NormalTok{)}
\end{Highlighting}
\end{Shaded}

\begin{table}[!h]
\centering
\caption{\label{tab:unnamed-chunk-35}EFA Loadings for MMB Items (2-Factor Solution)}
\centering
\begin{tabular}[t]{l>{\centering\arraybackslash}p{10cm}cc}
\toprule
\cellcolor[HTML]{f2f2f2}{\textbf{Item}} & \cellcolor[HTML]{f2f2f2}{\textbf{Label}} & \cellcolor[HTML]{f2f2f2}{\textbf{Factor1}} & \cellcolor[HTML]{f2f2f2}{\textbf{Factor2}}\\
\midrule
\cellcolor{white}{\textcolor{black}{MMBi1}} & \cellcolor{white}{\textcolor{black}{Allgemein verhalte ich mich männlich, weil ich die Akzeptanz und Anerkennung anderer möchte}} & \cellcolor{white}{\textcolor{black}{0.83}} & \cellcolor{white}{\textcolor{black}{}}\\
\cellcolor{white}{\textcolor{black}{MMBi2}} & \cellcolor{white}{\textcolor{black}{Allgemein bin ich männlich, weil das von mir erwartet wird}} & \cellcolor{white}{\textcolor{black}{0.9}} & \cellcolor{white}{\textcolor{black}{}}\\
\cellcolor{white}{\textcolor{black}{MMBi3}} & \cellcolor{white}{\textcolor{black}{Ich verhalte mich männlich, weil ich möchte, dass man mich mag}} & \cellcolor{white}{\textcolor{black}{0.79}} & \cellcolor{white}{\textcolor{black}{}}\\
\cellcolor{white}{\textcolor{black}{MMBi4}} & \cellcolor{white}{\textcolor{black}{Ich verhalte mich in Gegenwart anderer männlich, um ihre Erwartungen zu erfüllen}} & \cellcolor{white}{\textcolor{black}{0.89}} & \cellcolor{white}{\textcolor{black}{}}\\
\cellcolor{white}{\textcolor{black}{MMBi5}} & \cellcolor{white}{\textcolor{black}{Ich verhalte mich nicht weiblich, weil ich glaube, dass mich die Leute sonst nicht mögen würden}} & \cellcolor{white}{\textcolor{black}{0.78}} & \cellcolor{white}{\textcolor{black}{}}\\
\addlinespace
\cellcolor{white}{\textcolor{black}{MMBi6}} & \cellcolor{white}{\textcolor{black}{Es ist mir wichtig, männlich zu sein}} & \cellcolor{white}{\textcolor{black}{}} & \cellcolor{white}{\textcolor{black}{0.69}}\\
\cellcolor{white}{\textcolor{black}{MMBi7}} & \cellcolor{white}{\textcolor{black}{Ich bin gerne männlich}} & \cellcolor{white}{\textcolor{black}{}} & \cellcolor{white}{\textcolor{black}{0.74}}\\
\cellcolor{white}{\textcolor{black}{MMBi8}} & \cellcolor{white}{\textcolor{black}{Es macht mich glücklich, mich männlich zu verhalten}} & \cellcolor{white}{\textcolor{black}{}} & \cellcolor{white}{\textcolor{black}{0.8}}\\
\cellcolor{white}{\textcolor{black}{MMBi9}} & \cellcolor{white}{\textcolor{black}{Es ist mir wichtig, mich nicht weiblich zu verhalten}} & \cellcolor{white}{\textcolor{black}{0.39}} & \cellcolor{white}{\textcolor{black}{0.39}}\\
\bottomrule
\end{tabular}
\end{table}

\textbf{Interpretation}

\begin{itemize}
\tightlist
\item
  EFA sows a very distinct factor for pressured motivation
\item
  The factor for autonomous motivation is less clear and MMBi9 shows
  loadings on both factors
\end{itemize}

\subsubsection{4.1.5 Two-Factor Correlated
CFA}\label{two-factor-correlated-cfa}

\begin{itemize}
\tightlist
\item
  2 latent factors,
\item
  correlation allowed between factors
\end{itemize}

\begin{Shaded}
\begin{Highlighting}[]
\NormalTok{model\_twof\_cor }\OtherTok{\textless{}{-}} \StringTok{"}
\StringTok{  \# latent variables}
\StringTok{  Pressured =\textasciitilde{} MMBi1 + MMBi2 + MMBi3 + MMBi4 + MMBi5}
\StringTok{  Autonomous =\textasciitilde{} MMBi6 + MMBi7 + MMBi8 + MMBi9}

\StringTok{  \# allow correlation between factors}
\StringTok{  Pressured \textasciitilde{}\textasciitilde{} Autonomous}
\StringTok{"}

\NormalTok{results\_twof\_cor }\OtherTok{\textless{}{-}} \FunctionTok{run\_cfa\_model}\NormalTok{(model\_twof\_cor, df, }\AttributeTok{model\_name =} \StringTok{"Two{-}Factor Correlated"}\NormalTok{, }\AttributeTok{show\_mi =} \ConstantTok{TRUE}\NormalTok{)}
\end{Highlighting}
\end{Shaded}

\begin{verbatim}
## 
## -------------------------
## MODEL SUMMARY
## -------------------------
## lavaan 0.6.15 ended normally after 21 iterations
## 
##   Estimator                                         ML
##   Optimization method                           NLMINB
##   Number of model parameters                        19
## 
##   Number of observations                           154
## 
## Model Test User Model:
##                                                       
##   Test statistic                               113.810
##   Degrees of freedom                                26
##   P-value (Chi-square)                           0.000
## 
## Model Test Baseline Model:
## 
##   Test statistic                               923.528
##   Degrees of freedom                                36
##   P-value                                        0.000
## 
## User Model versus Baseline Model:
## 
##   Comparative Fit Index (CFI)                    0.901
##   Tucker-Lewis Index (TLI)                       0.863
## 
## Loglikelihood and Information Criteria:
## 
##   Loglikelihood user model (H0)              -2294.487
##   Loglikelihood unrestricted model (H1)      -2237.582
##                                                       
##   Akaike (AIC)                                4626.974
##   Bayesian (BIC)                              4684.676
##   Sample-size adjusted Bayesian (SABIC)       4624.539
## 
## Root Mean Square Error of Approximation:
## 
##   RMSEA                                          0.148
##   90 Percent confidence interval - lower         0.121
##   90 Percent confidence interval - upper         0.176
##   P-value H_0: RMSEA <= 0.050                    0.000
##   P-value H_0: RMSEA >= 0.080                    1.000
## 
## Standardized Root Mean Square Residual:
## 
##   SRMR                                           0.099
## 
## Parameter Estimates:
## 
##   Standard errors                             Standard
##   Information                                 Expected
##   Information saturated (h1) model          Structured
## 
## Latent Variables:
##                    Estimate  Std.Err  z-value  P(>|z|)   Std.lv  Std.all
##   Pressured =~                                                          
##     MMBi1             1.537    0.115   13.329    0.000    1.537    0.867
##     MMBi2             1.474    0.115   12.868    0.000    1.474    0.848
##     MMBi3             1.460    0.109   13.421    0.000    1.460    0.870
##     MMBi4             1.541    0.111   13.857    0.000    1.541    0.887
##     MMBi5             1.244    0.117   10.599    0.000    1.244    0.745
##   Autonomous =~                                                         
##     MMBi6             1.524    0.128   11.898    0.000    1.524    0.856
##     MMBi7             0.790    0.120    6.593    0.000    0.790    0.533
##     MMBi8             1.229    0.126    9.787    0.000    1.229    0.736
##     MMBi9             1.161    0.137    8.494    0.000    1.161    0.658
## 
## Covariances:
##                    Estimate  Std.Err  z-value  P(>|z|)   Std.lv  Std.all
##   Pressured ~~                                                          
##     Autonomous        0.587    0.064    9.126    0.000    0.587    0.587
## 
## Variances:
##                    Estimate  Std.Err  z-value  P(>|z|)   Std.lv  Std.all
##    .MMBi1             0.782    0.114    6.830    0.000    0.782    0.249
##    .MMBi2             0.849    0.119    7.134    0.000    0.849    0.281
##    .MMBi3             0.682    0.101    6.762    0.000    0.682    0.242
##    .MMBi4             0.641    0.100    6.382    0.000    0.641    0.212
##    .MMBi5             1.244    0.156    7.989    0.000    1.244    0.446
##    .MMBi6             0.849    0.201    4.222    0.000    0.849    0.268
##    .MMBi7             1.568    0.193    8.123    0.000    1.568    0.715
##    .MMBi8             1.277    0.192    6.662    0.000    1.277    0.458
##    .MMBi9             1.766    0.236    7.477    0.000    1.766    0.567
##     Pressured         1.000                               1.000    1.000
##     Autonomous        1.000                               1.000    1.000
## 
## 
## -------------------------
## MODEL FIT INDICES
## -------------------------
##       Index Value Criteria
## cfi     cfi 0.901     0.95
## tli     tli 0.863     0.95
## rmsea rmsea 0.148     0.06
## srmr   srmr 0.099     0.08
## 
## -------------------------
## STANDARDIZED FACTOR LOADINGS
## -------------------------
##       Factor  Item Std_Loading
## 1  Pressured MMBi1       0.867
## 2  Pressured MMBi2       0.848
## 3  Pressured MMBi3       0.870
## 4  Pressured MMBi4       0.887
## 5  Pressured MMBi5       0.745
## 6 Autonomous MMBi6       0.856
## 7 Autonomous MMBi7       0.533
## 8 Autonomous MMBi8       0.736
## 9 Autonomous MMBi9       0.658
## 
## -------------------------
## TOP MODIFICATION INDICES
## -------------------------
##           lhs op   rhs     mi    epc sepc.lv sepc.all sepc.nox
## 60      MMBi5 ~~ MMBi9 31.894  0.750   0.750    0.506    0.506
## 64      MMBi7 ~~ MMBi8 24.886  0.717   0.717    0.506    0.506
## 28 Autonomous =~ MMBi3 15.682  0.434   0.434    0.259    0.259
## 23  Pressured =~ MMBi7 14.810 -0.560  -0.560   -0.378   -0.378
## 25  Pressured =~ MMBi9 13.735  0.610   0.610    0.345    0.345
\end{verbatim}

Interpretation

\begin{itemize}
\tightlist
\item
  Allowing for a correlation between latent factors creates a better fit
  but still not sufficient
\item
  Mod indices suggest A correlation between Item MMBi5 and MMBi9, This
  makes conceptually sense, because they both deal with anti femininity.
\end{itemize}

\subsubsection{4.1.6 Two-Factor Correlated + Residuals
CFA}\label{two-factor-correlated-residuals-cfa}

\begin{itemize}
\tightlist
\item
  2 latent factors
\item
  correlation allowed between factors
\item
  allow correlated residuals
\end{itemize}

\begin{Shaded}
\begin{Highlighting}[]
\NormalTok{model\_twof\_cor\_mod }\OtherTok{\textless{}{-}} \StringTok{"}
\StringTok{  \# latent variables}
\StringTok{  Pressured =\textasciitilde{} MMBi1 + MMBi2 + MMBi3 + MMBi4 + MMBi5}
\StringTok{  Autonomous =\textasciitilde{} MMBi6 + MMBi7 + MMBi8 + MMBi9}

\StringTok{  Pressured \textasciitilde{}\textasciitilde{} Autonomous}

\StringTok{  \# theory{-}based correlated residuals}
\StringTok{  MMBi3 \textasciitilde{}\textasciitilde{} MMBi4}
\StringTok{  MMBi5 \textasciitilde{}\textasciitilde{} MMBi9}
\StringTok{  MMBi7 \textasciitilde{}\textasciitilde{} MMBi8}
\StringTok{"}

\NormalTok{results\_twof\_cor\_mod }\OtherTok{\textless{}{-}} \FunctionTok{run\_cfa\_model}\NormalTok{(model\_twof\_cor\_mod, df, }\AttributeTok{model\_name =} \StringTok{"Two{-}Factor Correlated Modified"}\NormalTok{, }\AttributeTok{show\_mi =} \ConstantTok{TRUE}\NormalTok{)}
\end{Highlighting}
\end{Shaded}

\begin{verbatim}
## 
## -------------------------
## MODEL SUMMARY
## -------------------------
## lavaan 0.6.15 ended normally after 26 iterations
## 
##   Estimator                                         ML
##   Optimization method                           NLMINB
##   Number of model parameters                        22
## 
##   Number of observations                           154
## 
## Model Test User Model:
##                                                       
##   Test statistic                                49.124
##   Degrees of freedom                                23
##   P-value (Chi-square)                           0.001
## 
## Model Test Baseline Model:
## 
##   Test statistic                               923.528
##   Degrees of freedom                                36
##   P-value                                        0.000
## 
## User Model versus Baseline Model:
## 
##   Comparative Fit Index (CFI)                    0.971
##   Tucker-Lewis Index (TLI)                       0.954
## 
## Loglikelihood and Information Criteria:
## 
##   Loglikelihood user model (H0)              -2262.144
##   Loglikelihood unrestricted model (H1)      -2237.582
##                                                       
##   Akaike (AIC)                                4568.288
##   Bayesian (BIC)                              4635.101
##   Sample-size adjusted Bayesian (SABIC)       4565.468
## 
## Root Mean Square Error of Approximation:
## 
##   RMSEA                                          0.086
##   90 Percent confidence interval - lower         0.052
##   90 Percent confidence interval - upper         0.119
##   P-value H_0: RMSEA <= 0.050                    0.041
##   P-value H_0: RMSEA >= 0.080                    0.642
## 
## Standardized Root Mean Square Residual:
## 
##   SRMR                                           0.074
## 
## Parameter Estimates:
## 
##   Standard errors                             Standard
##   Information                                 Expected
##   Information saturated (h1) model          Structured
## 
## Latent Variables:
##                    Estimate  Std.Err  z-value  P(>|z|)   Std.lv  Std.all
##   Pressured =~                                                          
##     MMBi1             1.502    0.116   12.902    0.000    1.502    0.847
##     MMBi2             1.450    0.115   12.608    0.000    1.450    0.835
##     MMBi3             1.513    0.108   14.031    0.000    1.513    0.902
##     MMBi4             1.590    0.110   14.413    0.000    1.590    0.915
##     MMBi5             1.154    0.111   10.378    0.000    1.154    0.711
##   Autonomous =~                                                         
##     MMBi6             1.547    0.129   12.023    0.000    1.547    0.869
##     MMBi7             0.690    0.123    5.585    0.000    0.690    0.466
##     MMBi8             1.165    0.128    9.136    0.000    1.165    0.698
##     MMBi9             1.211    0.132    9.200    0.000    1.211    0.679
## 
## Covariances:
##                    Estimate  Std.Err  z-value  P(>|z|)   Std.lv  Std.all
##   Pressured ~~                                                          
##     Autonomous        0.615    0.060   10.175    0.000    0.615    0.615
##  .MMBi3 ~~                                                              
##    .MMBi4            -0.228    0.078   -2.928    0.003   -0.228   -0.451
##  .MMBi5 ~~                                                              
##    .MMBi9             0.759    0.146    5.202    0.000    0.759    0.508
##  .MMBi7 ~~                                                              
##    .MMBi8             0.618    0.158    3.905    0.000    0.618    0.395
## 
## Variances:
##                    Estimate  Std.Err  z-value  P(>|z|)   Std.lv  Std.all
##    .MMBi1             0.887    0.120    7.379    0.000    0.887    0.282
##    .MMBi2             0.917    0.121    7.555    0.000    0.917    0.303
##    .MMBi3             0.523    0.104    5.027    0.000    0.523    0.186
##    .MMBi4             0.489    0.105    4.644    0.000    0.489    0.162
##    .MMBi5             1.305    0.157    8.323    0.000    1.305    0.495
##    .MMBi6             0.778    0.209    3.720    0.000    0.778    0.245
##    .MMBi7             1.715    0.208    8.263    0.000    1.715    0.783
##    .MMBi8             1.430    0.201    7.102    0.000    1.430    0.513
##    .MMBi9             1.710    0.237    7.212    0.000    1.710    0.539
##     Pressured         1.000                               1.000    1.000
##     Autonomous        1.000                               1.000    1.000
## 
## 
## -------------------------
## MODEL FIT INDICES
## -------------------------
##       Index Value Criteria
## cfi     cfi 0.971     0.95
## tli     tli 0.954     0.95
## rmsea rmsea 0.086     0.06
## srmr   srmr 0.074     0.08
## 
## -------------------------
## STANDARDIZED FACTOR LOADINGS
## -------------------------
##       Factor  Item Std_Loading
## 1  Pressured MMBi1       0.847
## 2  Pressured MMBi2       0.835
## 3  Pressured MMBi3       0.902
## 4  Pressured MMBi4       0.915
## 5  Pressured MMBi5       0.711
## 6 Autonomous MMBi6       0.869
## 7 Autonomous MMBi7       0.466
## 8 Autonomous MMBi8       0.698
## 9 Autonomous MMBi9       0.679
## 
## -------------------------
## TOP MODIFICATION INDICES
## -------------------------
##           lhs op   rhs     mi    epc sepc.lv sepc.all sepc.nox
## 31 Autonomous =~ MMBi3 12.225  0.398   0.398    0.237    0.237
## 26  Pressured =~ MMBi7 10.161 -0.431  -0.431   -0.291   -0.291
## 30 Autonomous =~ MMBi2  9.785 -0.378  -0.378   -0.217   -0.217
## 28  Pressured =~ MMBi9  5.333  0.392   0.392    0.220    0.220
## 52      MMBi3 ~~ MMBi8  4.164  0.176   0.176    0.203    0.203
\end{verbatim}

\textbf{Interpretation}

\begin{itemize}
\tightlist
\item
  \ldots{}
\end{itemize}

\subsection{4.2 Reliability}\label{reliability}

\begin{Shaded}
\begin{Highlighting}[]
\NormalTok{fit\_twof\_cor\_mod }\OtherTok{\textless{}{-}} \FunctionTok{cfa}\NormalTok{(model\_twof\_cor\_mod, }\AttributeTok{data =}\NormalTok{ df, }\AttributeTok{std.lv =} \ConstantTok{TRUE}\NormalTok{)}
\NormalTok{rel }\OtherTok{\textless{}{-}} \FunctionTok{reliability}\NormalTok{(fit\_twof\_cor\_mod)}
\FunctionTok{print}\NormalTok{(rel)}
\end{Highlighting}
\end{Shaded}

\begin{verbatim}
##        Pressured Autonomous
## alpha  0.9249521  0.7899473
## omega  0.9341165  0.7559467
## omega2 0.9341165  0.7559467
## omega3 0.9141315  0.7697912
## avevar 0.7182846  0.5026560
\end{verbatim}

\textbf{Interpretation}

\begin{itemize}
\tightlist
\item
  Pressured scale shows an excellent reliability (\textgreater{} .9)
\item
  Autonomous scale shows an acceptable reliability (\textgreater{} .7)
  =\textgreater{} reliability is sufficient, so we use the proposed
  factor structure to compute factor scored for autonomous and pressured
  motivation
\end{itemize}

Omega values ≥ .70 (acceptable), ≥ .80 (good). =\textgreater{} Alpha
assumes tau-equivalence (all items same loading), which isn't true here,
so we trust omega more

\subsection{4.3 Latent Trait Scores}\label{latent-trait-scores}

\subsubsection{4.3.1 Compute Scores}\label{compute-scores}

Use regression scoring from the chosen model, to compute latent trait
scores

\begin{Shaded}
\begin{Highlighting}[]
\NormalTok{fs\_reg }\OtherTok{\textless{}{-}} \FunctionTok{lavPredict}\NormalTok{(fit\_twof\_cor\_mod, }\AttributeTok{method =} \StringTok{"regression"}\NormalTok{)}

\NormalTok{df}\SpecialCharTok{$}\NormalTok{regPressured }\OtherTok{\textless{}{-}} \FunctionTok{scale}\NormalTok{(fs\_reg[, }\StringTok{"Pressured"}\NormalTok{])}
\NormalTok{df}\SpecialCharTok{$}\NormalTok{regAutonomous }\OtherTok{\textless{}{-}} \FunctionTok{scale}\NormalTok{(fs\_reg[, }\StringTok{"Autonomous"}\NormalTok{])}
\end{Highlighting}
\end{Shaded}

Also compute Row means for visualizations

\begin{Shaded}
\begin{Highlighting}[]
\NormalTok{df}\SpecialCharTok{$}\NormalTok{rowmeansAutonomous }\OtherTok{\textless{}{-}} \FunctionTok{rowMeans}\NormalTok{(}
\NormalTok{  df[mmb\_autonomous\_items],}
  \AttributeTok{na.rm =} \ConstantTok{TRUE}
\NormalTok{)}

\NormalTok{df}\SpecialCharTok{$}\NormalTok{rowmeansPressured }\OtherTok{\textless{}{-}} \FunctionTok{rowMeans}\NormalTok{(}
\NormalTok{  df[mmb\_pressured\_items],}
  \AttributeTok{na.rm =} \ConstantTok{TRUE}
\NormalTok{)}
\end{Highlighting}
\end{Shaded}

\subsubsection{4.3.2 Visualization \&
Descriptives}\label{visualization-descriptives-1}

\begin{Shaded}
\begin{Highlighting}[]
\FunctionTok{summary\_mean\_sd}\NormalTok{(df, }\FunctionTok{c}\NormalTok{(}\StringTok{"rowmeansAutonomous"}\NormalTok{, }\StringTok{"rowmeansPressured"}\NormalTok{, }\StringTok{"regAutonomous"}\NormalTok{, }\StringTok{"regPressured"}\NormalTok{))}
\end{Highlighting}
\end{Shaded}

\begin{verbatim}
## # A tibble: 4 x 5
##   variable            mean    sd     n missing
##   <chr>              <dbl> <dbl> <dbl>   <dbl>
## 1 rowmeansAutonomous  4.23  1.32   154       0
## 2 rowmeansPressured   3.43  1.51   154       0
## 3 regAutonomous       0     1      154       0
## 4 regPressured        0     1      154       0
\end{verbatim}

\begin{Shaded}
\begin{Highlighting}[]
\NormalTok{rm\_aut\_hist }\OtherTok{\textless{}{-}} \FunctionTok{plotHist}\NormalTok{(}\StringTok{"rowmeansAutonomous"}\NormalTok{, }\FunctionTok{c}\NormalTok{(}\DecValTok{0}\NormalTok{, }\DecValTok{7}\NormalTok{))}
\NormalTok{rm\_pres\_hist }\OtherTok{\textless{}{-}} \FunctionTok{plotHist}\NormalTok{(}\StringTok{"rowmeansPressured"}\NormalTok{, }\FunctionTok{c}\NormalTok{(}\DecValTok{0}\NormalTok{, }\DecValTok{7}\NormalTok{))}
\NormalTok{reg\_aut\_hist }\OtherTok{\textless{}{-}} \FunctionTok{plotHist}\NormalTok{(}\StringTok{"regAutonomous"}\NormalTok{, }\FunctionTok{c}\NormalTok{(}\SpecialCharTok{{-}}\DecValTok{5}\NormalTok{, }\DecValTok{5}\NormalTok{))}
\NormalTok{reg\_pres\_hist }\OtherTok{\textless{}{-}} \FunctionTok{plotHist}\NormalTok{(}\StringTok{"regPressured"}\NormalTok{, }\FunctionTok{c}\NormalTok{(}\SpecialCharTok{{-}}\DecValTok{5}\NormalTok{, }\DecValTok{5}\NormalTok{))}

\FunctionTok{grid.arrange}\NormalTok{(rm\_aut\_hist, rm\_pres\_hist, reg\_aut\_hist, reg\_pres\_hist, }\AttributeTok{ncol =} \DecValTok{2}\NormalTok{)}
\end{Highlighting}
\end{Shaded}

\begin{center}\includegraphics{analysis-report_files/figure-latex/unnamed-chunk-42-1} \end{center}

-\textgreater{} regression scoring reduces measurement error at the
price of losing variance

\begin{Shaded}
\begin{Highlighting}[]
\NormalTok{rm\_box }\OtherTok{\textless{}{-}} \FunctionTok{plotBox}\NormalTok{(}\FunctionTok{c}\NormalTok{(}\StringTok{"rowmeansAutonomous"}\NormalTok{, }\StringTok{"rowmeansPressured"}\NormalTok{))}
\CommentTok{\# ggpar(rm\_box, ylim = c(0, 7))}

\NormalTok{reg\_box }\OtherTok{\textless{}{-}} \FunctionTok{plotBox}\NormalTok{(}\FunctionTok{c}\NormalTok{(}\StringTok{"regAutonomous"}\NormalTok{, }\StringTok{"regPressured"}\NormalTok{))}
\CommentTok{\# ggpar(rm\_box, ylim = c({-}5, 5))}


\FunctionTok{grid.arrange}\NormalTok{(}\FunctionTok{ggpar}\NormalTok{(rm\_box, }\AttributeTok{ylim =} \FunctionTok{c}\NormalTok{(}\DecValTok{0}\NormalTok{, }\DecValTok{7}\NormalTok{)), }\FunctionTok{ggpar}\NormalTok{(reg\_box, }\AttributeTok{ylim =} \FunctionTok{c}\NormalTok{(}\SpecialCharTok{{-}}\DecValTok{5}\NormalTok{, }\DecValTok{5}\NormalTok{)), }\AttributeTok{ncol =} \DecValTok{2}\NormalTok{)}
\end{Highlighting}
\end{Shaded}

\begin{center}\includegraphics{analysis-report_files/figure-latex/unnamed-chunk-43-1} \end{center}

\begin{Shaded}
\begin{Highlighting}[]
\FunctionTok{cor}\NormalTok{(df}\SpecialCharTok{$}\NormalTok{regAutonomous, df}\SpecialCharTok{$}\NormalTok{regPressured)}
\end{Highlighting}
\end{Shaded}

\begin{verbatim}
##           [,1]
## [1,] 0.6702902
\end{verbatim}

\begin{Shaded}
\begin{Highlighting}[]
\FunctionTok{cor}\NormalTok{(df}\SpecialCharTok{$}\NormalTok{rowmeansAutonomous, df}\SpecialCharTok{$}\NormalTok{rowmeansPressured)}
\end{Highlighting}
\end{Shaded}

\begin{verbatim}
## [1] 0.4914834
\end{verbatim}

\textbf{Interpretation}

\begin{itemize}
\tightlist
\item
  moderate to high correlation between pressured and autonomous factor
\item
  the correlation increases with the regression scores (because factors
  are modeled as correlative)
\end{itemize}

\section{5. Test Quality Criteria: Aggressive
Cognition}\label{test-quality-criteria-aggressive-cognition}

\emph{Objectives:}

\begin{itemize}
\tightlist
\item
  Assess the \emph{validity} and \emph{reliability} of the scale for
  aggressive cognition
\item
  Compute participant latent trait scores
\end{itemize}

\emph{Methodology:}

\begin{itemize}
\tightlist
\item
  Visualization \& Descriptives
\end{itemize}

\subsection{5.1 Construct Validity}\label{construct-validity-1}

\subsubsection{5.1.1 Visualization \&
Descriptives}\label{visualization-descriptives-2}

\begin{Shaded}
\begin{Highlighting}[]
\NormalTok{cor\_aggr }\OtherTok{\textless{}{-}}\NormalTok{ psych}\SpecialCharTok{::}\FunctionTok{tetrachoric}\NormalTok{( df[, WFCT\_aggression\_items])}\SpecialCharTok{$}\NormalTok{rho}
\end{Highlighting}
\end{Shaded}

\begin{Shaded}
\begin{Highlighting}[]
\NormalTok{corrplot}\SpecialCharTok{::}\FunctionTok{corrplot}\NormalTok{(cor\_aggr,}
  \AttributeTok{method =} \StringTok{"color"}\NormalTok{,}
  \AttributeTok{type =} \StringTok{"upper"}\NormalTok{,}
  \AttributeTok{order =} \StringTok{"original"}\NormalTok{,}
  \AttributeTok{tl.col =} \StringTok{"black"}\NormalTok{,}
  \AttributeTok{tl.srt =} \DecValTok{45}\NormalTok{,}
  \AttributeTok{diag =} \ConstantTok{FALSE}
\NormalTok{)}
\end{Highlighting}
\end{Shaded}

\begin{center}\includegraphics{analysis-report_files/figure-latex/unnamed-chunk-45-1} \end{center}

-\textgreater{} a tetrachoric correlation is used, because responses are
always binary (aggressive, vs.~non-aggressive response)

\textbf{Interpretation}

\begin{itemize}
\tightlist
\item
  The correlation plot shows no clear pattern between the items
\item
  If all items were to measure the same latent construct (aggressive
  cognition) we would expect (somewhat) positive correlations between
  all the items
\end{itemize}

\begin{Shaded}
\begin{Highlighting}[]
\FunctionTok{summary\_mean\_sd}\NormalTok{(df, WFCT\_aggression\_items)}
\end{Highlighting}
\end{Shaded}

\begin{verbatim}
## # A tibble: 10 x 5
##    variable   mean    sd     n missing
##    <chr>     <dbl> <dbl> <dbl>   <dbl>
##  1 WFCTagg1   0.05  0.21   146       8
##  2 WFCTagg2   0.13  0.34   150       4
##  3 WFCTagg3   0.07  0.26   149       5
##  4 WFCTagg4   0.08  0.27   148       6
##  5 WFCTagg5   0.03  0.18   151       3
##  6 WFCTagg6   0.51  0.5    152       2
##  7 WFCTagg7   0.1   0.3    143      11
##  8 WFCTagg8   0.41  0.49   140      14
##  9 WFCTagg9   0.23  0.42   142      12
## 10 WFCTagg10  0.64  0.48   141      13
\end{verbatim}

\textbf{Interpretation}

\begin{itemize}
\tightlist
\item
  \emph{Low Means:} On items 1, 2, 3, 4, 5, 7
\item
  \emph{Missing Values:} 10\% missings on items 7, 8, 9, 10
\end{itemize}

\subsubsection{5.1.2 Check sampling adequacy for Factor
Analysis}\label{check-sampling-adequacy-for-factor-analysis}

\begin{Shaded}
\begin{Highlighting}[]
\FunctionTok{KMO}\NormalTok{(cor\_aggr)}
\end{Highlighting}
\end{Shaded}

\begin{verbatim}
## Kaiser-Meyer-Olkin factor adequacy
## Call: KMO(r = cor_aggr)
## Overall MSA =  0.16
## MSA for each item = 
##  WFCTagg1  WFCTagg2  WFCTagg3  WFCTagg4  WFCTagg5  WFCTagg6  WFCTagg7  WFCTagg8 
##      0.20      0.14      0.14      0.13      0.32      0.17      0.11      0.10 
##  WFCTagg9 WFCTagg10 
##      0.12      0.50
\end{verbatim}

\begin{Shaded}
\begin{Highlighting}[]
\FunctionTok{cortest.bartlett}\NormalTok{(cor\_aggr, }\AttributeTok{n =} \FunctionTok{nrow}\NormalTok{(df))}
\end{Highlighting}
\end{Shaded}

\begin{verbatim}
## $chisq
## [1] 623.5155
## 
## $p.value
## [1] 7.520443e-103
## 
## $df
## [1] 45
\end{verbatim}

KMO (Kaiser-Meyer-Olkin) should be \textgreater{} 0.6 (ideally
\textgreater{} 0.8) Bartlett's test should be significant (p \textless{}
.05)

\textbf{Interpretation}

\begin{itemize}
\tightlist
\item
  The KMO is extremely low, indicating that the items are not suited for
  a factor analysis
\item
  This is because the items share very little variance (are not
  correlating)
\end{itemize}

\subsection{5.2 Reliability}\label{reliability-1}

\begin{Shaded}
\begin{Highlighting}[]
\NormalTok{psych}\SpecialCharTok{::}\FunctionTok{alpha}\NormalTok{(cor\_aggr, }\AttributeTok{check.keys =} \ConstantTok{FALSE}\NormalTok{)}
\end{Highlighting}
\end{Shaded}

\begin{verbatim}
## Some items ( WFCTagg7 ) were negatively correlated with the first principal component and 
## probably should be reversed.  
## To do this, run the function again with the 'check.keys=TRUE' option
\end{verbatim}

\begin{verbatim}
## 
## Reliability analysis   
## Call: psych::alpha(x = cor_aggr, check.keys = FALSE)
## 
##   raw_alpha std.alpha G6(smc) average_r  S/N median_r
##       0.36      0.36    0.78     0.053 0.55     0.01
## 
##     95% confidence boundaries 
##       lower alpha upper
## Feldt -0.46  0.36  0.81
## 
##  Reliability if an item is dropped:
##           raw_alpha std.alpha G6(smc) average_r  S/N var.r   med.r
## WFCTagg1       0.31      0.31    0.72     0.048 0.45 0.047  0.0025
## WFCTagg2       0.33      0.33    0.72     0.052 0.50 0.049  0.0180
## WFCTagg3       0.32      0.32    0.71     0.050 0.47 0.047 -0.0106
## WFCTagg4       0.38      0.38    0.68     0.063 0.61 0.043  0.0409
## WFCTagg5       0.11      0.11    0.67     0.014 0.12 0.043 -0.0197
## WFCTagg6       0.24      0.24    0.62     0.033 0.31 0.044 -0.0106
## WFCTagg7       0.47      0.47    0.67     0.091 0.90 0.040  0.0497
## WFCTagg8       0.34      0.34    0.65     0.055 0.52 0.047  0.0180
## WFCTagg9       0.46      0.46    0.72     0.086 0.84 0.042  0.0497
## WFCTagg10      0.24      0.24    0.76     0.034 0.32 0.047  0.0011
## 
##  Item statistics 
##               r   r.cor r.drop
## WFCTagg1  0.426  0.3756  0.180
## WFCTagg2  0.385  0.3209  0.134
## WFCTagg3  0.410  0.3519  0.162
## WFCTagg4  0.281  0.2471  0.022
## WFCTagg5  0.750  0.7454  0.595
## WFCTagg6  0.566  0.5742  0.347
## WFCTagg7  0.024 -0.0200 -0.230
## WFCTagg8  0.365  0.3440  0.111
## WFCTagg9  0.075  0.0081 -0.183
## WFCTagg10 0.557  0.4518  0.336
\end{verbatim}

\textbf{Interpretation}

\subsection{5.3 Latent Trait Scores}\label{latent-trait-scores-1}

\subsubsection{5.3.1 Compute Scores}\label{compute-scores-1}

\begin{Shaded}
\begin{Highlighting}[]
\NormalTok{df}\SpecialCharTok{$}\NormalTok{aggressiveWordCompletionScore }\OtherTok{\textless{}{-}} \FunctionTok{rowMeans}\NormalTok{(}
  \FunctionTok{replace}\NormalTok{(df[WFCT\_aggression\_items], }\FunctionTok{is.na}\NormalTok{(df[WFCT\_aggression\_items]), }\DecValTok{0}\NormalTok{)}
\NormalTok{)}

\NormalTok{df}\SpecialCharTok{$}\NormalTok{aggressiveWordCompletionScoreTransformed }\OtherTok{\textless{}{-}} \FunctionTok{asin}\NormalTok{(}\FunctionTok{sqrt}\NormalTok{(df}\SpecialCharTok{$}\NormalTok{aggressiveWordCompletionScore))}
\end{Highlighting}
\end{Shaded}

\subsubsection{5.3.2 Visualization \&
Descriptives}\label{visualization-descriptives-3}

\begin{Shaded}
\begin{Highlighting}[]
\FunctionTok{summary\_mean\_sd}\NormalTok{(df, }\FunctionTok{c}\NormalTok{(}\StringTok{"aggressiveWordCompletionScore"}\NormalTok{))}
\end{Highlighting}
\end{Shaded}

\begin{verbatim}
## # A tibble: 1 x 5
##   variable                       mean    sd     n missing
##   <chr>                         <dbl> <dbl> <dbl>   <dbl>
## 1 aggressiveWordCompletionScore  0.21  0.12   154       0
\end{verbatim}

\begin{Shaded}
\begin{Highlighting}[]
\FunctionTok{plotHist}\NormalTok{(}\StringTok{"aggressiveWordCompletionScore"}\NormalTok{, }\FunctionTok{c}\NormalTok{(}\DecValTok{0}\NormalTok{, }\DecValTok{1}\NormalTok{), }\AttributeTok{condition =} \ConstantTok{NULL}\NormalTok{, }\AttributeTok{binwidth =} \FloatTok{0.1}\NormalTok{)}
\end{Highlighting}
\end{Shaded}

\begin{center}\includegraphics{analysis-report_files/figure-latex/unnamed-chunk-51-1} \end{center}

\subsection{Summary}\label{summary}

\ldots{}

\section{6. Test Quality Criteria: Anxious
Cognition}\label{test-quality-criteria-anxious-cognition}

\emph{Objectives:}

\begin{itemize}
\tightlist
\item
  Assess the \emph{validity} and \emph{reliability} of the scale for
  anxious cognition
\item
  Compute participant latent trait scores
\end{itemize}

\emph{Methodology:}

\begin{itemize}
\tightlist
\item
  Descriptives \& Visualization
\end{itemize}

\subsection{6.1 Construct Validity}\label{construct-validity-2}

\subsubsection{6.1.1 Visualization \&
Descriptives}\label{visualization-descriptives-4}

\begin{Shaded}
\begin{Highlighting}[]
\NormalTok{cor\_anx }\OtherTok{\textless{}{-}}\NormalTok{ psych}\SpecialCharTok{::}\FunctionTok{tetrachoric}\NormalTok{(df[, WFCT\_anxiety\_items])}\SpecialCharTok{$}\NormalTok{rho}
\end{Highlighting}
\end{Shaded}

\begin{Shaded}
\begin{Highlighting}[]
\NormalTok{corrplot}\SpecialCharTok{::}\FunctionTok{corrplot}\NormalTok{(cor\_anx,}
  \AttributeTok{method =} \StringTok{"color"}\NormalTok{,}
  \AttributeTok{type =} \StringTok{"upper"}\NormalTok{,}
  \AttributeTok{order =} \StringTok{"original"}\NormalTok{,}
  \AttributeTok{tl.col =} \StringTok{"black"}\NormalTok{,}
  \AttributeTok{tl.srt =} \DecValTok{45}\NormalTok{,}
  \AttributeTok{diag =} \ConstantTok{FALSE}
\NormalTok{)}
\end{Highlighting}
\end{Shaded}

\begin{center}\includegraphics{analysis-report_files/figure-latex/unnamed-chunk-52-1} \end{center}

-\textgreater{} a tetrachoric correlation is used, because responses are
always binary (aggressive, vs.~non-aggressive)

\begin{Shaded}
\begin{Highlighting}[]
\FunctionTok{summary\_mean\_sd}\NormalTok{(df, WFCT\_anxiety\_items)}
\end{Highlighting}
\end{Shaded}

\begin{verbatim}
## # A tibble: 10 x 5
##    variable   mean    sd     n missing
##    <chr>     <dbl> <dbl> <dbl>   <dbl>
##  1 WFCTanx1   0.31  0.47   144      10
##  2 WFCTanx2   0.03  0.18   145       9
##  3 WFCTanx3   0.85  0.36   108      46
##  4 WFCTanx4   0.26  0.44   145       9
##  5 WFCTanx5   0.33  0.47   140      14
##  6 WFCTanx6   0.36  0.48   148       6
##  7 WFCTanx7   0.16  0.37   144      10
##  8 WFCTanx8   0.08  0.27   151       3
##  9 WFCTanx9   0.86  0.35   105      49
## 10 WFCTanx10  0.17  0.37   144      10
\end{verbatim}

\textbf{Interpretation}

\begin{itemize}
\tightlist
\item
  \emph{Low Means:}
\item
  \emph{Missing Values:} \ldots{}
\end{itemize}

\textbf{Interpretation}

\begin{itemize}
\tightlist
\item
  \ldots{}
\end{itemize}

\subsubsection{6.1.2 Check sampling adequacy for Factor
Analysis}\label{check-sampling-adequacy-for-factor-analysis-1}

\begin{Shaded}
\begin{Highlighting}[]
\FunctionTok{KMO}\NormalTok{(cor\_anx)}
\end{Highlighting}
\end{Shaded}

\begin{verbatim}
## Kaiser-Meyer-Olkin factor adequacy
## Call: KMO(r = cor_anx)
## Overall MSA =  0.3
## MSA for each item = 
##  WFCTanx1  WFCTanx2  WFCTanx3  WFCTanx4  WFCTanx5  WFCTanx6  WFCTanx7  WFCTanx8 
##      0.48      0.35      0.22      0.53      0.27      0.47      0.43      0.22 
##  WFCTanx9 WFCTanx10 
##      0.26      0.27
\end{verbatim}

\begin{Shaded}
\begin{Highlighting}[]
\FunctionTok{cortest.bartlett}\NormalTok{(cor\_anx, }\AttributeTok{n =} \FunctionTok{nrow}\NormalTok{(df))}
\end{Highlighting}
\end{Shaded}

\begin{verbatim}
## $chisq
## [1] 432.2395
## 
## $p.value
## [1] 1.010417e-64
## 
## $df
## [1] 45
\end{verbatim}

KMO (Kaiser-Meyer-Olkin) should be \textgreater{} 0.6 (ideally
\textgreater{} 0.8) Bartlett's test should be significant (p \textless{}
.05)

\textbf{Interpretation}

\begin{itemize}
\tightlist
\item
  The KMO is extremely low, indicating that the items are not suited for
  a factor analysis
\item
  This is because the items share very little variance \ldots{}
\end{itemize}

\subsection{6.2 Reliability}\label{reliability-2}

\begin{Shaded}
\begin{Highlighting}[]
\NormalTok{psych}\SpecialCharTok{::}\FunctionTok{alpha}\NormalTok{(cor\_anx, }\AttributeTok{check.keys =} \ConstantTok{FALSE}\NormalTok{)}
\end{Highlighting}
\end{Shaded}

\begin{verbatim}
## Some items ( WFCTanx2 WFCTanx7 WFCTanx8 ) were negatively correlated with the first principal component and 
## probably should be reversed.  
## To do this, run the function again with the 'check.keys=TRUE' option
\end{verbatim}

\begin{verbatim}
## 
## Reliability analysis   
## Call: psych::alpha(x = cor_anx, check.keys = FALSE)
## 
##   raw_alpha std.alpha G6(smc) average_r   S/N median_r
##      0.086     0.086    0.51    0.0093 0.094   0.0075
## 
##     95% confidence boundaries 
##       lower alpha upper
## Feldt -1.08  0.09  0.73
## 
##  Reliability if an item is dropped:
##           raw_alpha std.alpha G6(smc) average_r    S/N var.r   med.r
## WFCTanx1     -0.075    -0.075   0.405   -0.0078 -0.069 0.043 -0.0051
## WFCTanx2      0.195     0.195   0.548    0.0263  0.243 0.047  0.0345
## WFCTanx3      0.058     0.058   0.338    0.0068  0.062 0.042  0.0090
## WFCTanx4      0.222     0.222   0.566    0.0307  0.285 0.045  0.0589
## WFCTanx5      0.145     0.145   0.487    0.0185  0.170 0.045  0.0074
## WFCTanx6      0.029     0.029   0.509    0.0033  0.030 0.052 -0.0051
## WFCTanx7      0.146     0.146   0.531    0.0186  0.171 0.047  0.0329
## WFCTanx8      0.235     0.235   0.494    0.0330  0.307 0.043  0.0589
## WFCTanx9     -0.403    -0.403   0.063   -0.0329 -0.287 0.037 -0.0644
## WFCTanx10    -0.034    -0.034   0.386   -0.0036 -0.032 0.045 -0.0050
## 
##  Item statistics 
##               r  r.cor r.drop
## WFCTanx1  0.516  0.480  0.240
## WFCTanx2  0.144 -0.081 -0.160
## WFCTanx3  0.356  0.381  0.056
## WFCTanx4  0.095 -0.145 -0.206
## WFCTanx5  0.228  0.105 -0.077
## WFCTanx6  0.394  0.195  0.098
## WFCTanx7  0.227  0.039 -0.079
## WFCTanx8  0.070 -0.073 -0.228
## WFCTanx9  0.791  0.991  0.623
## WFCTanx10 0.470  0.461  0.185
\end{verbatim}

\subsection{6.3 Latent Trait Scores}\label{latent-trait-scores-2}

\subsubsection{6.3.1 Compute Scores}\label{compute-scores-2}

\begin{Shaded}
\begin{Highlighting}[]
\NormalTok{df}\SpecialCharTok{$}\NormalTok{anxiousWordCompletionScore }\OtherTok{\textless{}{-}} \FunctionTok{rowMeans}\NormalTok{(}
  \FunctionTok{replace}\NormalTok{(df[WFCT\_anxiety\_items], }\FunctionTok{is.na}\NormalTok{(df[WFCT\_anxiety\_items]), }\DecValTok{0}\NormalTok{)}
\NormalTok{)}

\NormalTok{df}\SpecialCharTok{$}\NormalTok{anxiousWordCompletionScoreTransformed }\OtherTok{\textless{}{-}} \FunctionTok{asin}\NormalTok{(}\FunctionTok{sqrt}\NormalTok{(df}\SpecialCharTok{$}\NormalTok{anxiousWordCompletionScore))}
\end{Highlighting}
\end{Shaded}

\subsubsection{6.3.2 Visualization \&
Descriptives}\label{visualization-descriptives-5}

\begin{Shaded}
\begin{Highlighting}[]
\FunctionTok{summary\_mean\_sd}\NormalTok{(df, }\FunctionTok{c}\NormalTok{(}\StringTok{"anxiousWordCompletionScore"}\NormalTok{))}
\end{Highlighting}
\end{Shaded}

\begin{verbatim}
## # A tibble: 1 x 5
##   variable                    mean    sd     n missing
##   <chr>                      <dbl> <dbl> <dbl>   <dbl>
## 1 anxiousWordCompletionScore  0.28  0.14   154       0
\end{verbatim}

\begin{Shaded}
\begin{Highlighting}[]
\FunctionTok{plotHist}\NormalTok{(}\StringTok{"anxiousWordCompletionScore"}\NormalTok{, }\FunctionTok{c}\NormalTok{(}\DecValTok{0}\NormalTok{, }\DecValTok{1}\NormalTok{), }\AttributeTok{condition =} \ConstantTok{NULL}\NormalTok{, }\AttributeTok{binwidth =} \FloatTok{0.1}\NormalTok{)}
\end{Highlighting}
\end{Shaded}

\begin{center}\includegraphics{analysis-report_files/figure-latex/unnamed-chunk-58-1} \end{center}

\subsection{Summary}\label{summary-1}

\section{Hypothesis 1a \& 1b}\label{hypothesis-1a-1b}

\subsection{Visualization}\label{visualization}

Aggressive and anxious word completions by threat condition

\begin{Shaded}
\begin{Highlighting}[]
\FunctionTok{plotBox}\NormalTok{(}
  \FunctionTok{c}\NormalTok{(}\StringTok{"aggressiveWordCompletionScore"}\NormalTok{, }\StringTok{"anxiousWordCompletionScore"}\NormalTok{),}
  \StringTok{"threatCondition"}
\NormalTok{)}
\end{Highlighting}
\end{Shaded}

\begin{center}\includegraphics{analysis-report_files/figure-latex/unnamed-chunk-59-1} \end{center}

\begin{Shaded}
\begin{Highlighting}[]
\FunctionTok{summary\_mean\_sd}\NormalTok{(df, }\FunctionTok{c}\NormalTok{(}\StringTok{"aggressiveWordCompletionScore"}\NormalTok{, }\StringTok{"anxiousWordCompletionScore"}\NormalTok{), }\StringTok{"threatCondition"}\NormalTok{)}
\end{Highlighting}
\end{Shaded}

\begin{verbatim}
## # A tibble: 4 x 6
##   threatCondition variable                       mean    sd     n missing
##   <fct>           <chr>                         <dbl> <dbl> <dbl>   <dbl>
## 1 threat          aggressiveWordCompletionScore  0.19  0.12    75       0
## 2 threat          anxiousWordCompletionScore     0.28  0.13    75       0
## 3 noThreat        aggressiveWordCompletionScore  0.23  0.12    79       0
## 4 noThreat        anxiousWordCompletionScore     0.28  0.15    79       0
\end{verbatim}

\subsection{Hypothesis 1a t-test}\label{hypothesis-1a-t-test}

\subsubsection{Assumption 1: Normality withing each
group}\label{assumption-1-normality-withing-each-group}

Check Visually

\begin{Shaded}
\begin{Highlighting}[]
\FunctionTok{check\_normality\_qq}\NormalTok{(}
\NormalTok{  df, aggressiveWordCompletionScoreTransformed,}
\NormalTok{  threatCondition}
\NormalTok{)}
\end{Highlighting}
\end{Shaded}

\begin{center}\includegraphics{analysis-report_files/figure-latex/unnamed-chunk-61-1} \end{center}

-\textgreater{} data points should be roughly on the line

\begin{Shaded}
\begin{Highlighting}[]
\FunctionTok{check\_normality\_hist}\NormalTok{(}
\NormalTok{  df, aggressiveWordCompletionScoreTransformed,}
\NormalTok{  threatCondition}
\NormalTok{)}
\end{Highlighting}
\end{Shaded}

\begin{center}\includegraphics{analysis-report_files/figure-latex/unnamed-chunk-62-1} \end{center}

-\textgreater{} should roughly resemble a normal distribution

\subsubsection{Assumption 2: Homogeneity of
variance}\label{assumption-2-homogeneity-of-variance}

Levene Test

\begin{Shaded}
\begin{Highlighting}[]
\FunctionTok{check\_homogeneity}\NormalTok{(}
\NormalTok{  df, aggressiveWordCompletionScoreTransformed,}
\NormalTok{  threatCondition}
\NormalTok{)}
\end{Highlighting}
\end{Shaded}

\begin{verbatim}
## Levene's Test for Homogeneity of Variance (center = median)
##        Df F value Pr(>F)
## group   1  0.1237 0.7256
##       152
\end{verbatim}

p \textgreater{} .05 → equal variances (use Student's t-test: var.equal
= TRUE)

\begin{Shaded}
\begin{Highlighting}[]
\FunctionTok{summary\_mean\_sd}\NormalTok{(df, }\StringTok{"aggressiveWordCompletionScore"}\NormalTok{, }\AttributeTok{group =} \StringTok{"threatCondition"}\NormalTok{)}
\end{Highlighting}
\end{Shaded}

\begin{verbatim}
## # A tibble: 2 x 6
##   threatCondition variable                       mean    sd     n missing
##   <fct>           <chr>                         <dbl> <dbl> <dbl>   <dbl>
## 1 threat          aggressiveWordCompletionScore  0.19  0.12    75       0
## 2 noThreat        aggressiveWordCompletionScore  0.23  0.12    79       0
\end{verbatim}

\begin{Shaded}
\begin{Highlighting}[]
\FunctionTok{run\_ttest}\NormalTok{(df, aggressiveWordCompletionScoreTransformed, threatCondition,}
  \AttributeTok{alternative =} \StringTok{"greater"}\NormalTok{, }\AttributeTok{var\_equal =} \ConstantTok{TRUE}
\NormalTok{)}
\end{Highlighting}
\end{Shaded}

\begin{verbatim}
## 
##  Two Sample t-test
## 
## data:  aggressiveWordCompletionScoreTransformed by threatCondition
## t = -1.9812, df = 152, p-value = 0.9753
## alternative hypothesis: true difference in means between group threat and group noThreat is greater than 0
## 95 percent confidence interval:
##  -0.1097734        Inf
## sample estimates:
##   mean in group threat mean in group noThreat 
##              0.4174215              0.4772327
\end{verbatim}

\subsection{Hypothesis 1b t-test}\label{hypothesis-1b-t-test}

\subsubsection{Assumption 1: Normality withing each
group}\label{assumption-1-normality-withing-each-group-1}

Check Visually

\begin{Shaded}
\begin{Highlighting}[]
\FunctionTok{check\_normality\_qq}\NormalTok{(df, anxiousWordCompletionScoreTransformed, threatCondition)}
\end{Highlighting}
\end{Shaded}

\begin{center}\includegraphics{analysis-report_files/figure-latex/unnamed-chunk-66-1} \end{center}

-\textgreater{} data points should be roughly on the line

\begin{Shaded}
\begin{Highlighting}[]
\FunctionTok{check\_normality\_hist}\NormalTok{(}
\NormalTok{  df, anxiousWordCompletionScoreTransformed,}
\NormalTok{  threatCondition}
\NormalTok{)}
\end{Highlighting}
\end{Shaded}

\begin{center}\includegraphics{analysis-report_files/figure-latex/unnamed-chunk-67-1} \end{center}

-\textgreater{} should roughly resemble a normal distribution

\subsubsection{Assumption 2: Homogeneity of
variance}\label{assumption-2-homogeneity-of-variance-1}

Levene Test

\begin{Shaded}
\begin{Highlighting}[]
\FunctionTok{check\_homogeneity}\NormalTok{(df, anxiousWordCompletionScoreTransformed, threatCondition)}
\end{Highlighting}
\end{Shaded}

\begin{verbatim}
## Levene's Test for Homogeneity of Variance (center = median)
##        Df F value Pr(>F)
## group   1    0.65 0.4214
##       152
\end{verbatim}

p \textgreater{} .05 → equal variances (use Student's t-test: var.equal
= TRUE)

\begin{Shaded}
\begin{Highlighting}[]
\FunctionTok{summary\_mean\_sd}\NormalTok{(df, }\StringTok{"anxiousWordCompletionScore"}\NormalTok{, }\AttributeTok{group =} \StringTok{"threatCondition"}\NormalTok{)}
\end{Highlighting}
\end{Shaded}

\begin{verbatim}
## # A tibble: 2 x 6
##   threatCondition variable                    mean    sd     n missing
##   <fct>           <chr>                      <dbl> <dbl> <dbl>   <dbl>
## 1 threat          anxiousWordCompletionScore  0.28  0.13    75       0
## 2 noThreat        anxiousWordCompletionScore  0.28  0.15    79       0
\end{verbatim}

Run inference statistic

\begin{Shaded}
\begin{Highlighting}[]
\FunctionTok{run\_ttest}\NormalTok{(df, anxiousWordCompletionScoreTransformed, threatCondition,}
  \AttributeTok{alternative =} \StringTok{"greater"}\NormalTok{, }\AttributeTok{var\_equal =} \ConstantTok{TRUE}
\NormalTok{)}
\end{Highlighting}
\end{Shaded}

\begin{verbatim}
## 
##  Two Sample t-test
## 
## data:  anxiousWordCompletionScoreTransformed by threatCondition
## t = 0.29844, df = 152, p-value = 0.3829
## alternative hypothesis: true difference in means between group threat and group noThreat is greater than 0
## 95 percent confidence interval:
##  -0.0401006        Inf
## sample estimates:
##   mean in group threat mean in group noThreat 
##              0.5392036              0.5303812
\end{verbatim}

\section{Hypothesis 2a \& 2b: Moderation of Threat
Response}\label{hypothesis-2a-2b-moderation-of-threat-response}

\subsection{Visualization}\label{visualization-1}

2a: Moderation autonomous motivation for masculine behavior on
aggressive word completions

\begin{Shaded}
\begin{Highlighting}[]
\FunctionTok{plotLine}\NormalTok{(}\StringTok{"regAutonomous"}\NormalTok{, }\StringTok{"anxiousWordCompletionScoreTransformed"}\NormalTok{,}
  \AttributeTok{condition =} \StringTok{"threatCondition"}\NormalTok{, }\AttributeTok{df =}\NormalTok{ df}
\NormalTok{)}
\end{Highlighting}
\end{Shaded}

\begin{center}\includegraphics{analysis-report_files/figure-latex/unnamed-chunk-71-1} \end{center}

2b: Moderation pressured motivation for masculine behavior on aggressive
word completions

\begin{Shaded}
\begin{Highlighting}[]
\FunctionTok{plotLine}\NormalTok{(}\StringTok{"regPressured"}\NormalTok{, }\StringTok{"aggressiveWordCompletionScoreTransformed"}\NormalTok{,}
  \AttributeTok{condition =} \StringTok{"threatCondition"}\NormalTok{, }\AttributeTok{df =}\NormalTok{ df}
\NormalTok{)}
\end{Highlighting}
\end{Shaded}

\begin{center}\includegraphics{analysis-report_files/figure-latex/unnamed-chunk-72-1} \end{center}

\subsection{Hypothesis 2a}\label{hypothesis-2a}

\subsubsection{Multiple linear
regression}\label{multiple-linear-regression}

Establish model

\begin{Shaded}
\begin{Highlighting}[]
\NormalTok{model\_anxious }\OtherTok{\textless{}{-}} \FunctionTok{lm}\NormalTok{(anxiousWordCompletionScoreTransformed }\SpecialCharTok{\textasciitilde{}}
\NormalTok{  threatCondition }\SpecialCharTok{*}\NormalTok{ regAutonomous, }\AttributeTok{data =}\NormalTok{ df)}
\end{Highlighting}
\end{Shaded}

\paragraph{Assumption 1: Normality}\label{assumption-1-normality}

\begin{Shaded}
\begin{Highlighting}[]
\FunctionTok{plot}\NormalTok{(model\_anxious, }\AttributeTok{which =} \DecValTok{2}\NormalTok{)}
\end{Highlighting}
\end{Shaded}

\begin{center}\includegraphics{analysis-report_files/figure-latex/unnamed-chunk-74-1} \end{center}

-\textgreater{} Q-Q plot points roughly on the line → residuals
\textasciitilde{} normal. -\textgreater{} identify possible outliers
(highly influential points)

\paragraph{Assumption 2: Outliers}\label{assumption-2-outliers}

Examine outliers via residuals

\begin{Shaded}
\begin{Highlighting}[]
\NormalTok{res }\OtherTok{\textless{}{-}} \FunctionTok{residuals}\NormalTok{(model\_anxious)}

\CommentTok{\# Standardized residuals}
\NormalTok{res\_std }\OtherTok{\textless{}{-}} \FunctionTok{rstandard}\NormalTok{(model\_anxious)}

\CommentTok{\# Flag extreme residuals}
\NormalTok{outliers }\OtherTok{\textless{}{-}} \FunctionTok{which}\NormalTok{(}\FunctionTok{abs}\NormalTok{(res\_std) }\SpecialCharTok{\textgreater{}} \DecValTok{2}\NormalTok{)}
\NormalTok{outliers}
\end{Highlighting}
\end{Shaded}

\begin{verbatim}
##  35  53  73  96 115 128 141 
##  35  53  73  96 115 128 141
\end{verbatim}

Examine outliers via cook's distance

\begin{Shaded}
\begin{Highlighting}[]
\NormalTok{influential\_obs }\OtherTok{\textless{}{-}}\NormalTok{ df}
\NormalTok{influential\_obs}\SpecialCharTok{$}\NormalTok{cooksd }\OtherTok{\textless{}{-}} \FunctionTok{cooks.distance}\NormalTok{(model\_anxious)}

\NormalTok{n }\OtherTok{\textless{}{-}} \FunctionTok{nrow}\NormalTok{(df)}
\NormalTok{k }\OtherTok{\textless{}{-}} \FunctionTok{length}\NormalTok{(}\FunctionTok{coef}\NormalTok{(model\_anxious)) }\SpecialCharTok{{-}} \DecValTok{1}
\NormalTok{threshold }\OtherTok{\textless{}{-}} \DecValTok{4} \SpecialCharTok{/}\NormalTok{ (n }\SpecialCharTok{{-}}\NormalTok{ k }\SpecialCharTok{{-}} \DecValTok{1}\NormalTok{)}


\CommentTok{\# influential\_obs \%\textgreater{}\%}
\CommentTok{\# filter(cooksd \textgreater{} threshold)}
\end{Highlighting}
\end{Shaded}

Cook's distance \textgreater{} 4/(n-k-1) → potentially influential

\paragraph{Assumption 3: No
Multikollinearity}\label{assumption-3-no-multikollinearity}

\begin{Shaded}
\begin{Highlighting}[]
\FunctionTok{cor}\NormalTok{(df}\SpecialCharTok{$}\NormalTok{regAutonomous, }\FunctionTok{as.numeric}\NormalTok{(}\FunctionTok{factor}\NormalTok{(df}\SpecialCharTok{$}\NormalTok{threatCondition,}
  \AttributeTok{levels =} \FunctionTok{c}\NormalTok{(}\StringTok{"noThreat"}\NormalTok{, }\StringTok{"threat"}\NormalTok{)}
\NormalTok{)) }\SpecialCharTok{{-}} \DecValTok{1}\NormalTok{)}
\end{Highlighting}
\end{Shaded}

\begin{verbatim}
##             [,1]
## [1,] -0.01320326
\end{verbatim}

-\textgreater{} would also make no sense, since I randomly assign
threatCondition

Inference statistics

\begin{Shaded}
\begin{Highlighting}[]
\FunctionTok{summary}\NormalTok{(model\_anxious)}
\end{Highlighting}
\end{Shaded}

\begin{verbatim}
## 
## Call:
## lm(formula = anxiousWordCompletionScoreTransformed ~ threatCondition * 
##     regAutonomous, data = df)
## 
## Residuals:
##      Min       1Q   Median       3Q      Max 
## -0.54610 -0.07455  0.04050  0.14365  0.35547 
## 
## Coefficients:
##                                        Estimate Std. Error t value Pr(>|t|)    
## (Intercept)                            0.539281   0.021311  25.305   <2e-16 ***
## threatConditionnoThreat               -0.008895   0.029754  -0.299    0.765    
## regAutonomous                          0.005748   0.022312   0.258    0.797    
## threatConditionnoThreat:regAutonomous -0.006172   0.030009  -0.206    0.837    
## ---
## Signif. codes:  0 '***' 0.001 '**' 0.01 '*' 0.05 '.' 0.1 ' ' 1
## 
## Residual standard error: 0.1845 on 150 degrees of freedom
## Multiple R-squared:  0.001031,   Adjusted R-squared:  -0.01895 
## F-statistic: 0.05158 on 3 and 150 DF,  p-value: 0.9845
\end{verbatim}

\section{Post Hoc Analysis}\label{post-hoc-analysis}

\ldots{}

\end{document}
